\documentclass[../poma-notes.tex]{subfiles}

\begin{document}

函数与其相关的术语在 Definition 2.1 与 2.2 中有过介绍。尽管我们更应该(在之后的章节中)关注实数与复数函数(例如函数中
值为实数或复数),同样讨论向量化值的函数(例如函数的值在 $R^k$ 中)以及在任意的度量空间中的函数。在这些范畴内讨论的定理
并不会比仅约束在实数函数中讨论更容易,这例如实际上简化了以及区分了场景,以抛弃不必要的假设并在更适当且普遍的上下文中陈述
以及证明定理。

函数定义的领域同样是在度量空间上,适用于不同特化的实例。

\subsection*{Limits of Functions}

\begin{definition}
  Let $X$ and $Y$ be metric spaces; suppose $E \subset X$, $f$ maps $E$ into $Y$, and $p$ is a limit
  point of $E$. We write $f(x) \to q$ as $x \to p$, or
  \begin{equation}
    \lim_{x \to p} f(x) = q
  \end{equation}
  if there is a point $q \in Y$ with the following property: For every $\varepsilon > 0$ there exists a
  $\delta > 0$ such that
  \begin{equation}
    d_Y(f(x),q) < \varepsilon
  \end{equation}
  for all points $x \in E$ for which
  \begin{equation}
    0 < d_X(x,p) < \delta
  \end{equation}
\end{definition}

此处的 $d_X$ 与 $d_Y$ 分别代表 $X$ 与 $Y$ 中的距离。

如果 $X$ 以及/或者 $Y$ 被替换成实数线,复数平面,或者是欧式空间 $R^k$,那么距离 $d_X, d_Y$ 也被替换为绝对值,或者是距离
(详见 Sec 2.16)。

需要注意的是 $p \in X$,这里的 $p$ 需要是上述定义中,非 $E$ 上的任何一点。除此之外,即使如果 $p \in E$,我们很容易得到
$f(p) \ne \lim_{x \to p} f(x)$。

\begin{anote}
  \textbf{函数的极限}:令 $f: E \subset X \to Y$,$X, Y$ 为度量空间,且 $p$ 是 $E$ 的极限点。凡是当 $x \to p$ 时
  $f(x) \to q$,或 $\lim_{x \to p} f(x) = q$ 时,存在 $q \in Y$ 具有以下性质:对每个 $\varepsilon > 0$,存在
  $\delta > 0$,使 $d_Y(f(x),q) < \varepsilon$ 对满足 $0 < d_X(x, p) < \delta$ 的一切点 $x \in E$ 成立。
\end{anote}

\newpage
我们可以将上述定义重新表现为\underline{序列的极限形式}:

\begin{theorem}
  Let $X, Y, E, f$, and $p$ be as in Definition 4.1. Then
  \begin{equation}
    \lim_{x \to p} f(x) = q
  \end{equation}
  if and only if
  \begin{equation}
    \lim_{n \to \infty} f(p_n) = q
  \end{equation}
  for every sequence $\{p_n\}$ in $E$ such that
  \begin{equation}
    p_n \ne p, \qquad \lim_{n \to \infty} p_n = p
  \end{equation}
\end{theorem}

\begin{proof}
  假设 (4) 成立。在 $E$ 中选择 $\{p_n\}$ 满足 (6)。令 $\varepsilon > 0$,那么如果 $x\in E$ 以及 $0<d_X(x,p)<\varepsilon$
  存在 $\delta > 0$ 使得 $d_Y(f(x),q)<\varepsilon$。同样的,存在 $N$ 使得 $n > N$ 有 $0<d_X(p_n,p)<\delta$。因此,
  对于 $n > N$ 有 $d_Y(f(p_n),q) < \varepsilon$,即证明了 (5) 成立。

  相反的,假设 (4) 不成立。那么存在某些 $\varepsilon > 0$,对于每个 $\delta > 0$ 存在一个点 $x \in E$(依赖 $\delta$),
  满足 $d_Y(f(x),q)\ge\varepsilon$ 但是 $0<d_X(x,p)<\delta$。选取 $\delta_n = 1/n\ (n=1,2,3,\dots)$,因此
  找到一个在 $E$ 中的序列满足 (6) 而 (5) 却不成立。
\end{proof}

\begin{corollary}
  If $f$ has a limit at $p$, this limit is unique.
\end{corollary}

\begin{anote}
  如果 $f$ 在 $p$ 点有极限,那么它是唯一的。
\end{anote}

\begin{definition}
  Suppose we have two complex functions, $f$ and $g$, both defined on $E$. By $f + g$ we mean the function which
  assigns to each point $x$ of $E$ the number $f(x) + g(x)$. Similarly we define the difference $f - g$, the
  product $fg$, and the quotient $f/g$ of the two functions, with the understanding that the quotient is defined
  only at those points $x$ of $E$ at which $g(x) \ne 0$. If $f$ assigns to each point $x$ of $E$ the same number
  $c$, then $f$ is said to be a constant functions, and if $f(x) \ge g(x)$ for every $x \in E$, we shall sometimes
  write $f \ge g$, for brevity.

  Similarly, if $\mathbf{f}$ and $\mathbf{g}$ map $E$ into $R^k$, we define $\mathbf{f}+\mathbf{g}$ and
  $\mathbf{f}\cdot\mathbf{g}$ by
  \[
    (\mathbf{f}+\mathbf{g})(x)=\mathbf{f}(x)+\mathbf{g}(x),\qquad
    (\mathbf{f}\cdot\mathbf{g})(x)=\mathbf{f}(x)\cdot\mathbf{g}(x);
  \]
  and if $\lambda$ is a real number, $(\lambda\mathbf{f}(x))=\lambda\mathbf{f}(x)$.
\end{definition}

\begin{theorem}
  Suppose $E \subset X$, a metric space, $p$ is a limit point of $E$, $f$ and $g$ are complex functions on $E$, and
  \[
    \lim_{x \to p} f(x) = A, \qquad \lim_{x \to p} g(x) = B.
  \]
  Then
  \begin{enumerate}[label=(\alph*)]
    \item $\lim_{x \to p} (f+g)(x) = A + B$;
    \item $\lim_{x \to p} (fg)(x) = AB$
    \item $\lim_{x \to p} \Bigl(\frac{f}{g}\Bigr)(x) = \frac{A}{B}, \text{if } B \ne 0$.
  \end{enumerate}
\end{theorem}

\begin{proof}
  根据 Theorem 4.2,上述声明遵循了类似序列的属性(Theorem 3.3)。
\end{proof}

\begin{remark*}
  如果 $f$ 与 $g$ 映射 $E$ 至 $R^k$,那么 (a) 仍然为真,而 (b) 则变为
  (b$'$) $\ \lim_{x \to p}(\mathbf{f}\cdot\mathbf{g})(x) = \mathbf{A}\cdot\mathbf{B}$。
\end{remark*}

\end{document}
