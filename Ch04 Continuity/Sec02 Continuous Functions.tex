\documentclass[../poma-notes.tex]{subfiles}

\begin{document}

\subsection*{Continuous Functions}

\begin{definition}
  Suppose $X$ and $Y$ are metric spaces, $E \subset X,\ p \in E$, and $f$ maps $E$ into $Y$. Then $f$ is said
  to be \textit{continuous at $p$} if for every $\varepsilon > 0$ there exists a $\delta > 0$ such that
  \[
    d_Y(f(x),f(p)) < \varepsilon
  \]
  for all points $x \in E$ for which $d_X(x,p) < \delta$.

  If $f$ is continuous at every point of $E$, then $f$ is said to be \textit{continuous on $E$}.
\end{definition}

需要注意的是 $f$ 必须定义在点 $p$ 上才可在 $p$ 上连续(将此与 Definition 4.1 后的 remark 进行比较)。

如果 $p$ 是 $E$ 上的独立点,那么通过上述定义可得:所有拥有 $E$ 作为其领域的函数 $f$ 在点 $p$ 上连续。无论选取何值 $\varepsilon>0$,
可以选取 $\delta > 0$ 使得 $x \in E$ 对于 $d_X(x,p) < \delta$ 而言有 $x = p$;那么
\[
  d_Y(f(x),f(p)) = 0 < \delta
\]

\begin{theorem}
  In the situation given in Definition 4.5, assume also that $p$ is a limit point of $E$. Then $f$ is continuous
  at $p$ if and only if $\lim_{x \to p} f(x) = f(p)$.
\end{theorem}

\begin{proof}
  将 Definition 4.1 与 4.5 相比较即可得出结论。
\end{proof}

\begin{anote}
  \textbf{连续性}:$\lim_{x \to p} f(x) = f(p)$
\end{anote}

现在开始函数组合的部分。以下定理的一个简单声明:一个连续函数的函数若是连续的,那么该函数仍然是连续的。

\begin{theorem}
  Suppose $X, Y, Z$ are metric spaces, $E \subset X$, $f$ maps $E$ into $Y$, $g$ maps the range of $f, f(E)$,
  into $Z$, and $h$ is the mapping of $E$ into $Z$ defined by
  \[
    h(x) = g(f(x)) \qquad (x \in E).
  \]
  If $f$ is continuous at a point $p \in E$ and if $g$ is continuous at the point $f(p)$, then $h$ is continuous
  at $p$.

  This function $h$ is called the \textit{composition} or the \textit{composite} of $f$ and $g$. The notation
  \[
    h = g \circ f
  \]
  is frequently used in this context.
\end{theorem}

\begin{proof}
  给定 $\varepsilon > 0$。因为 $g$ 在 $f(p)$ 上连续,存在 $\eta > 0$ 使得
  \[
    d_Z(g(y),g(f(p))) < \varepsilon \text{ if } d_Y(y,f(p)) < \eta \text{ and } y \in f(E).
  \]
  因为 $f$ 在 $p$ 上连续,存在 $\delta > 0$ 使得
  \[
    d_Y(f(x),f(p)) < \eta \text{ if } d_X(x,p) < \delta \text{ and } x \in E.
  \]
  如果 $d_X(x,p) < \varepsilon$ 以及 $x \in E$,那么就有
  \[
    d_Z(h(x),h(p)) = d_Z(g(f(x)), g(f(p))) < \varepsilon
  \]
  因此 $h$ 在 $p$ 上连续。
\end{proof}

\begin{theorem}
  A mapping $f$ of a metric space $X$ into a metric space $Y$ is continuous on $X$ if and only if $f^{-1}(V)$
  is open in $X$ for every open set $V$ in $Y$.
\end{theorem}

(相反的镜像定义于 Definition 2.2)对于连续性而言,这是一个非常有用的特性。

\begin{proof}
  假设 $f$ 在 $X$ 上连续,同时 $V$ 是 $Y$ 上的一个开集。我们需要展示 $f^{-1}(V)$ 的每个点都是 $f^{-1}(V)$ 的一个内点。
  假设 $p \in X$ 以及 $f(p) \in V$。由于 $V$ 为开,存在 $\varepsilon > 0$ 满足 $y \in V$ 如果 $d_Y(f(p),y)<\varepsilon$;
  又因为 $f$ 在 $p$ 上连续,存在 $\delta > 0$ 满足 $d_Y(f(x),f(p))<\varepsilon$ 如果 $d_X(x,p)<\delta$。因此一旦
  $d_X(x,p)<\delta$ 则有 $x \in f^{-1}(V)$。

  相反的,假设对于在 $Y$ 上的每个开集 $V$ 有 $f^{-1}(V)$ 在 $X$ 上为开。固定 $p \in X$ 以及 $\varepsilon > 0$,令 $V$
  为所有 $y \in Y$ 的集合,满足 $d_Y(y,f(p))<\varepsilon$。那么 $V$ 为开;因此 $f^{-1}(V)$ 为开;因此存在 $\delta > 0$
  一旦 $d_X(p,x) < \delta$ 则有 $x \in f^{-1}(V)$。但是若 $x \in f^{-1}(V)$,那么 $f(x) \in V$,使得
  $d_Y(f(x),f(p))<\varepsilon$。

  证明完成。
\end{proof}

\begin{anote}
  度量空间 $X, Y$ 的函数 $f: X \to Y$ 连续,当且仅当对 $Y$ 的每个开集 $V$,$f^{-1}(V)$ 是 $X$ 中的开集。
  这个是连续性的一个极为有用的特征。
\end{anote}

\begin{corollary}
  A mapping $f$ of a metric space $X$ into a metric space $Y$ is continuous if and only if $f^{-1}(C)$ is closed
  in $X$ for every closed set $C$ in $Y$.
\end{corollary}

此条推论遵循了上述定理,因为一个集为闭,当且仅当其补集为开,且因为对于每个 $E \subset Y$ 有 $f^{-1}(E^c) = [f^{-1}(E)]^c$。

\begin{anote}
  \textbf{推论}:度量空间 $X, Y$ 的函数 $f: X \to Y$ 连续,当且仅当对 $Y$ 的每个闭集 $C$,$f^{-1}(C)$ 是 $X$ 中的闭集。
\end{anote}

现在更进一步到复数与向量函数,以及定义在 $R^k$ 上的子集。

\begin{theorem}
  Let $f$ and $g$ be complex continuous functions on a metric space $X$. Then $f + g$, $fg$, and $f/g$ are
  continuous on $X$.
\end{theorem}

最后一个情况,需要假设 $g(x) \ne 0$,对于所有 $x \in X$。

\begin{proof}
  若是 $X$ 的独立点,则无需进行证明。对于极限点,此声明遵循 Theorems 4.4 与 4.6。
\end{proof}

\begin{theorem}\mbox{}\par
  \begin{enumerate}[label=(\alph*)]
    \item Let $f_1, \dots, f_k$ be real functions on a metric space $X$, and let $\pmb{f}$ be the mapping of
          $X$ into $R^k$ defined by
          \begin{equation}
            \mathbf{f}(x) = (f_1(x),\dots,f_k(x)) \qquad (x \in X);
          \end{equation}
          then $\mathbf{f}$ is continuous if and only if each of the functions $f_1,\dots,f_k$ is continuous.
    \item If $\mathbf{f}$ and $\mathbf{g}$ are continuous mappings of $X$ into $R^k$, then $\mathbf{f + g}$
          and $\mathbf{f \cdot g}$ are continuous on $X$.
  \end{enumerate}
\end{theorem}

函数 $f_1,\dots,f_k$ 被称为 $\mathbf{f}$ 的 \textit{分量 components} 函数。注意 $\mathbf{f + g}$ 是一个至 $R^k$ 的映射,
而 $\mathbf{f \cdot g}$ 则是在 $X$ 上的 \textit{实 real} 函数。

\begin{proof}
  对于 $j = 1,\dots,k$ 而言,(a) 遵循不等式
  \[
    |f_j(x) - f_j(y)| \le |\mathbf{f}(x) - \mathbf{f}(y)| =
    \biggl\{ \sum_{i=1}^k |f_i(x) - f_i(y)|^2 \biggr\}^{\frac{1}{2}}
  \]
  而 (b) 遵循 (a) 以及 Theorem 4.9。
\end{proof}

% TODO

\end{document}

