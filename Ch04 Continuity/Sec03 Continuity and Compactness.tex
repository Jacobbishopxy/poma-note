\documentclass[../poma-notes.tex]{subfiles}

\begin{document}

\subsection*{Continuity and Compactness}

\begin{definition}
  A mapping $\mathbf{f}$ of a set $E$ into $R^k$ is said to be \textit{bounded} if there is a real number
  $M$ such that $|\mathbf{f}(x)| \le M$ for all $x \in E$.
\end{definition}

\begin{anote}
  \textbf{有界}:$\mathbf{f}(x) \le M$。
\end{anote}

\begin{theorem}
  Suppose $f$ is a continuous mapping of a compact metric space $X$ into a metric space $Y$. Then $f(X)$ is
  compact.
\end{theorem}

\begin{proof}
  令 $\{V_\alpha\}$ 为 $f(X)$ 的一个开覆盖。由于 $f$ 是连续的,Theorem 4.8 展示了每个 $f^{-1}(V_\alpha)$ 集合皆为开。
  由于 $X$ 是紧的,拥有有限个索引,假设 $\alpha_1,\dots,\alpha_n$ 满足
  \begin{equation}
    X \subset f^{-1}(V_{\alpha_1}) \cup \cdots \cup f^{-1}(V_{\alpha_n})
  \end{equation}
  又因为对于任何 $E \subset Y$ 有 $f(f^{-1}(E)) \subset E$,(12) 说明
  \begin{equation}
    f(X) \subset V_{\alpha_1} \cup \cdots \cup V_{\alpha_n}
  \end{equation}

  证明完毕。
\end{proof}

注意:这里使用了 $f(f^{-1}(E)) \subset E$ 的关系,对于 $E \subset Y$ 有效。如果是 $E \subset X$,那么则是
$f^{-1}(f(E)) \supset E$;等式在任何一种情况下都不需要相等。

\begin{anote}
  若紧度量空间 $X$ 到度量空间 $Y$ 的函数 $f: X \to Y$ 是连续的,那么 $f(X)$ 是紧的。
\end{anote}

现在来推导一下 Theorem 4.14 的某些结果。

\begin{theorem}
  If $\mathbf{f}$ is a continuous mapping of a compact metric space $X$ into $R^k$, then $\mathbf{f}(X)$
  is closed and bounded. Thus, $\mathbf{f}$ is bounded.
\end{theorem}

上述定理遵循了 Theorem 2.41。当 $f$ 为实(real)时,其结果特别重要:

\begin{anote}
  如果 $\mathbf{f}$ 是把紧度量空间 $X$ 映入 $R^k$ 内的连续映射,那么 $\mathbf{f}(X)$ 是闭合有界的。因此 $\mathbf{f}$
  是有界的。
\end{anote}

\begin{theorem}
  Suppose $f$ is a continuous real function on a compact metric space $X$, and
  \begin{equation}
    M = \sup_{p \in X} f(p), \qquad m = \inf_{p \in X} f(p).
  \end{equation}
  Then there exist points $p, q \in X$ such that $f(p) = M$ and $f(q) = m$.
\end{theorem}

式 (14) 意味着 $M$ 是集合中所有数 $f(p)$ 的最小上界,而 $p$ 的范围超过 $X$,而 $m$ 则是这些数的最大下界。

该结论同样也可以这样表述:\textit{在 $X$ 中存在点 $p$ 与 $q$ 对所有 $x \in X$ 有 $f(q) \le f(x) \le f(p)$};即,$f$
(在 $p$) 达到最大,(在 $q$)达到最小。

\begin{proof}
  根据 Theorem 4.15,$f(X)$ 是一个闭且有界的实数集合;因此根据 Theorem 2.28,$f(X)$ 包含
  \[
    M = \sup f(X) \qquad \text{and} \qquad m = \inf f(X)
  \]
\end{proof}

\begin{anote}
  如果 $\mathbf{f}$ 是紧度量空间 $X$ 上的连续实函数,且 $M = \sup_{p \in X} f(p),\ m = \inf_{p \in X} f(p)$,
  那么一定存在 $r,\ s \in X$ 使 $f(r) = M$ 以及 $f(x) = m$。
\end{anote}

\begin{theorem}
  Suppose $f$ is a continuous 1-1 mapping of a compact metric space $X$ onto a metric space $Y$. Then the
  inverse mapping $f^{-1}$ defined on $Y$ by
  \[
    f^{-1}(f(x)) = x \qquad (x \in X)
  \]
  is a continuous mapping of $Y$ onto $X$.
\end{theorem}

\begin{proof}
  将 Theorem 4.8 应用至 $f^{-1}$ 代替 $f$,足够证明对于每个在 $X$ 中的开集 $V$ 而言,$f(V)$ 在 $Y$ 上是一个开集。

  $V$ 的补集 $V^c$ 在 $X$ 中为闭,因此是紧的(Theorem 2.35);因此 $f(V^c)$ 是一个 $Y$ 上的紧子集(Theorem 4.14)
  并在 $Y$ 上为闭(Theorem 2.34)。因为 $f$ 是一对一的映射,$f(V)$ 是 $f(V^c)$ 的补集。因此 $f(V)$ 是开的。
\end{proof}

\begin{anote}
  设 $f$ 是把紧度量空间 $X$ 映满度量空间 $Y$ 的连续 1-1 映射,那么逆映射 $f^{-1}$ 是 $Y$ 映满 $X$ 的连续映射。
\end{anote}

\begin{definition}
  Let $f$ be a mapping of a metric space $X$ into a metric space $Y$. We say that $f$ is \textit{uniformly
    continuous} on $X$ if for every $\varepsilon > 0$ there exists $\delta > 0$ such that
  \begin{equation}
    f_Y(f(p),f(q)) < \varepsilon
  \end{equation}
  for all $p$ and $q$ in $X$ for which $d_X(p,q) < \delta$.
\end{definition}

\begin{anote}
  对度量空间 $X, Y$ 的函数 $f: X \to Y$,称 $f$ 在 $X$ 上一致连续,若对每个 $\varepsilon > 0$ 总存在 $\delta > 0$
  对一切满足 $d_X(p,q) < \delta$ 的 $p, q \in X$ 都能使 $d_Y(f(p), f(q)) < \varepsilon$。
\end{anote}

让我们来讨论一下连续性与一致连续性概念的区别。首先,一致连续性是函数在集合上的属性,且连续性可以被定义在单一一个点上。在某个
点上讨论一致连续性是无意义的。其次,如果 $f$ 在 $X$ 上连续,那么可以找到,对于每个 $\varepsilon > 0$ 以及对于每个 $X$
上的 $p$,具有 Definition 4.5 所定义的 $\delta > 0$。该 $\delta$ 依赖 $\varepsilon$ 与 $p$。而如果 $f$ 在 $X$
上一致连续,那么对于每个 $\varepsilon > 0$,找到\textit{一个} $\delta > 0$ 满足\textit{所有} $X$ 的点 $p$ 是可能的。

显然的,每个一致连续函数都是连续的。而根据下面的定理,可知这两个概念在紧集上是相等的。

\begin{theorem}
  Let $f$ be a continuous mapping of a compact metric space $X$ into a metric space $Y$. Then $f$ is
  uniformly continuous on $X$.
\end{theorem}

\begin{proof}
  给定 $\varepsilon > 0$。由于 $f$ 是连续的,我们可以为每个点 $p \in X$ 关联一个正数 $\phi(p)$ 满足
  \begin{equation}
    q \in X, d_X(p,q) < \phi(p) \quad \text{implies} \quad d_Y(f(p),f(q)) < \frac{\varepsilon}{2}
  \end{equation}
  令 $J(p)$ 为所有 $q \in X$ 的集合,有
  \begin{equation}
    d_X(p,q) < \frac{1}{2} \phi(p)
  \end{equation}
  因为 $p \in J(p)$,所有 $J(p)$ 集合的集合群是 $X$ 的一个开覆盖;而又因为 $X$ 是紧的,在 $X$ 中存在点 $p_1, \dots, p_n$
  的一个有限集合满足
  \begin{equation}
    X \subset J(p_1) \cup \cdots \cup J(p_n)
  \end{equation}

  令
  \begin{equation}
    \delta = \frac{1}{2} \min [\phi(p_1),\dots,\phi(p_n)]
  \end{equation}
  那么有 $\delta > 0$。(这是从紧凑性定义中继承的重要一点:覆盖的有限性。有限正数集的最小值是正数,而无限正数集的 inf 很可能为 0。)

  现在领 $q$ 与 $p$ 为 $X$ 上的点,满足 $d_X(p,q)<\delta$。根据 (18),存在一个整数 $m, 1 \le m \le n$,满足 $p \in J(p_n)$;
  因此
  \begin{equation}
    d_X(p, p_m) < \frac{1}{2} \phi(p_m)
  \end{equation}
  同样有
  \[
    d_X(q, p_m) \le d_X(p,q) + d_X(p,p_m) < \delta + \frac{1}{2} \phi(p_m) \le \phi(p_m)
  \]
  最后,(16) 可得
  \[
    d_Y(f(p),f(q)) \le d_Y(f(p),f(p_m)) + d_Y(f(q),f(p_m)) < \varepsilon
  \]
  证明完毕。
\end{proof}

\begin{anote}
  设 $f$ 是把紧度量空间 $X$ 映入度量空间 $Y$ 的连续映射,那么 $f$ 在 $X$ 上一致连续。
\end{anote}

\begin{theorem}
  Let $E$ be a non-compact set in $R^1$. Then
  \begin{enumerate}[label=(\alph*)]
    \item there exists a continuous function on $E$ which is not bounded;
    \item there exists a continuous and bounded function on $E$ which has no maximum.
  \end{enumerate}

  If, in addition, $E$ is bounded, then
  \begin{enumerate}
    \item [(c)] there exists a continuous function on $E$ which is not uniformly continuous.
  \end{enumerate}
\end{theorem}

% TODO

\end{document}

