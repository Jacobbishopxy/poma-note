\PassOptionsToPackage{quiet}{fontspec}
\documentclass[UTF8]{ctexart}

% ================================================================================================
% Dependencies
% ================================================================================================

\usepackage[a4paper, total={6in, 8in}]{geometry}
\usepackage{subfiles}
\usepackage{anyfontsize}
\usepackage{amsfonts}
\usepackage{xcolor}
\usepackage{graphicx}
\usepackage{listings}
\usepackage{xcolor}
\usepackage{amsthm,amssymb}
\usepackage[leqno]{amsmath}
\usepackage[inline]{enumitem}
\usepackage{xpatch}
\usepackage{hyperref}
\usepackage{tikz}
\usepackage{pgfplots}
\usepackage{mathrsfs}
\usepackage{dirtytalk}

% ================================================================================================
% math graphics setting
% ================================================================================================

\pgfplotsset{compat=1.15}
\usetikzlibrary{arrows,matrix,hobby,shapes}

% ================================================================================================
% setting theorem counter & style
% ================================================================================================

% theorem styles
\newtheoremstyle{thm}{}{}{\itshape}{}{\bfseries}{}{.5em}{{\thmnumber{#2 }}{\thmname{#1}}{\thmnote{ (#3)}}}
\newtheoremstyle{dfn}{}{}{}{}{\bfseries}{}{.5em}{{\thmnumber{#2 }}{\thmname{#1}}{\thmnote{ (#3)}}}
\newtheoremstyle{named}{}{}{\itshape}{}{\bfseries}{}{.5em}{{\thmnumber{#2 }}{\thmnote{#3}}}

% theorem counter
\newtheorem{poma}{PoMA}[section]

\theoremstyle{thm}
\newtheorem{theorem}[poma]{Theorem}
\newtheorem*{corollary}{Corollary}
\newtheorem{proposition}[poma]{proposition}

\theoremstyle{dfn}
\newtheorem{remark}[poma]{Remark}
\newtheorem{remarks}[poma]{Remarks}
\newtheorem{example}[poma]{Example}
\newtheorem{examples}[poma]{Examples}
\newtheorem{definition}[poma]{Definition}
\newtheorem{definitions}[poma]{Definitions}
\newtheorem{notation}[poma]{Notation}

\theoremstyle{named}
\newtheorem{namedtheorem}[poma]{}

% ================================================================================================
% import & settings
% ================================================================================================

% img folder import
\graphicspath{\subfix{./images/}}
% style file import
\usepackage{./style}
\lstset{style=mystyle}

% ================================================================================================
% miscellaneous
% ================================================================================================

% custom proof style

% a new command which severs arbitrary section number
\newcommand{\asection}[2]{
    \setcounter{section}{#1}
    \addtocounter{section}{-1}
    \section{#2}
}

% a new command which simplify \lstinline
\newcommand{\acode}[1]{
    \lstinline[basicstyle=\ttfamily]{#1}
}

% a new command which denotes "note"
\newcommand{\anote}{
    \paragraph*{Note}
}

\renewcommand*{\proofname}{Proof}
\xapptocmd{\proof}{\mbox{}\par\nobreak}{}{}

% equation number alignment
\makeatletter % ams classes trick
\def\fullwidthdisplay{\displayindent\z@ \displaywidth\columnwidth}
\edef\@tempa{\noexpand\fullwidthdisplay\the\everydisplay}
\everydisplay\expandafter{\@tempa}
\makeatother

% reset equation number after each section
\counterwithin*{equation}{section}

% ================================================================================================
% content
% ================================================================================================

\title{Study Notes of Principles of Mathematical Analysis}
\author{Jacob Xie}
\date{January 18, 2023}

\begin{document}

\maketitle
\newpage

\section{The Real and Complex Number Systems}
\subfile{Ch01 The Real and Complex Number Systems/The Real and Complex Number Systems}
\newpage

\section{Basic Topology}
\subfile{Ch02 Basic Topology/Sec01 Finite, Countable, and Uncountable Sets}
\subfile{Ch02 Basic Topology/Sec02 Compact Sets}
\subfile{Ch02 Basic Topology/Sec03 Perfect Sets}
\subfile{Ch02 Basic Topology/Sec04 Connected Sets}
\newpage

\section{Numerical Sequences and Series}
\subfile{Ch03 Numerical Sequences and Series/Sec01 Convergent Sequences}
\subfile{Ch03 Numerical Sequences and Series/Sec02 Subsequences}
\subfile{Ch03 Numerical Sequences and Series/Sec03 Cauchy Sequences}
\subfile{Ch03 Numerical Sequences and Series/Sec04 Upper and Lower Limits}
\subfile{Ch03 Numerical Sequences and Series/Sec05 Some Special Sequences}
% \subfile{Ch03 Numerical Sequences and Series/Sec06 Series}
% \subfile{Ch03 Numerical Sequences and Series/Sec07 Series of Nonnegative Terms}
% \subfile{Ch03 Numerical Sequences and Series/Sec08 The Number e}
% \subfile{Ch03 Numerical Sequences and Series/Sec09 The Root and Ratio Tests}
% \subfile{Ch03 Numerical Sequences and Series/Sec10 Power Series}
% \subfile{Ch03 Numerical Sequences and Series/Sec11 Summation by Parts}
% \subfile{Ch03 Numerical Sequences and Series/Sec12 Absolute Convergence}
% \subfile{Ch03 Numerical Sequences and Series/Sec13 Addition and Multiplication of Series}
% \subfile{Ch03 Numerical Sequences and Series/Sec14 Rearrangements}
\newpage

\section{Continuity}
% \subfile{}
\newpage

\section{Differentiation}
% \subfile{}
\newpage

\section{The Riemann-Stiltjes Integral}
% \subfile{}
\newpage

\section{Sequences and Series of Functions}
% \subfile{}
\newpage

\section{Some Special Functions}
% \subfile{}
\newpage

\section{Functions of Several Variables}
% \subfile{}
\newpage

\section{Integration of Differential Forms}
% \subfile{}
\newpage

\section{The Lebesgue Theory}
% \subfile{}
\newpage

\end{document}
