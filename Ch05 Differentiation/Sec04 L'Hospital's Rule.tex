\documentclass[../poma-notes.tex]{subfiles}

\begin{document}

\newpage
\subsection*{L'Hospital's Rule}

以下的定理在极限的计算中使用非常频繁。

\begin{theorem}
  Suppose $f$ and $g$ are real and differentiable in $(a, b)$, and $g'(x) \ne 0$ for all $x \in (a,b)$, where
  $-\infty \le a < b \le +\infty$. Suppose
  \begin{equation}
    \frac{f'(x)}{g'(x)} \to A \ \text{as} \ x \to a.
  \end{equation}
  If
  \begin{equation}
    f(x) \to 0 \ \text{and} \ g(x) \to 0 \ \text{as} \ x \to a,
  \end{equation}
  or if
  \begin{equation}
    g(x) \to +\infty \ \text{as} \ x \to a,
  \end{equation}
  then
  \begin{equation}
    \frac{f(x)}{g(x)} \to A \ \text{as} \ x \to a.
  \end{equation}
\end{theorem}

类似的声明当然也是成立的,如果 $x \to b$,或是 (15) 的 $g(x) \to -\infty$。注意我们现在使用的极限概念是 Definition 4.33 中的
派生理念。

\begin{proof}
  我们首先考虑 $-\infty \le A < +\infty$ 的情况。选择一个实数 $q$ 满足 $A < q$,接着选择 $r$ 满足 $A < r < q$。根据 (13)
  存在一个点 $c \in (a,b)$ 满足 $a < x < c$ 得
  \begin{equation}
    \frac{f'(x)}{g'(x)} < r
  \end{equation}
  如果 $a < x < y < c$,那么 Theorem 5.9 可知存在一个点 $t \in (x,y)$ 满足
  \begin{equation}
    \frac{f(x) - f(y)}{g(x) - g(y)} = \frac{f'(t)}{g'(t)} < r
  \end{equation}

  假设 (14) 成立。令 $x \to a$ 如 (18),可得
  \begin{equation}
    \frac{f(y)}{g(y)} \le r < q \qquad (a<y<c)
  \end{equation}

  接着,假设 (15) 成立。保持如 (18) 中的 $y$ 固定,我们选择一个点 $c_1 \in (a,y)$ 满足 $g(x) > g(y)$ 以及如果 $a<x<c_1$ 有
  $g(x)$。将 (18) 乘以 $[g(x) - g(y)]/g(x)$,可得
  \begin{equation}
    \frac{f(x)}{g(x)} < r - r \frac{g(y)}{g(x)} + \frac{f(y)}{g(x)} \qquad (a<x<c_1)
  \end{equation}

  如果如 (20) 令 $x \to a$,(15) 可知存在一个点 $c_2 \in (a, c_1)$ 满足
  \begin{equation}
    \frac{f(x)}{g(x)} < q \qquad (a < x < c_2)
  \end{equation}

  总结,(19) 与 (21) 可知对于任何 $q$,仅售条件限制 $A < q$,存在一个点 $c_2$ 如果 $a < x < c_2$ 满足 $f(x)/g(x) < q$。

  以同样的方式,如果 $-\infty < A \le +\infty$,已经选定 $p$ 满足 $p < A$,我们可以找到一个点 $c_3$ 满足
  \begin{equation}
    p < \frac{f(x)}{g(x)} \qquad (a<x<c_3)
  \end{equation}
  同时 (16) 遵循上述两个声明。
\end{proof}

\begin{anote}
  洛必达法则:假设实函数 $f,\ g$ 在 $(a, b)$ 内可微,而且对所有 $x \in (a,b),\ g'(x) \ne 0$,其中 $-\infty\le a<b \le+\infty$。
  已知 $\lim_{x\to a}f'(x)/g'(x)=A$,如果 $\lim_{x\to a}f(x)=\lim_{x\to a} g(x) = 0$,或者 $\lim_{x\to a}g(x)=+\infty$,
  那么 $\lim_{x\to a} f(x)/g(x) = A$。

  下列四个条件对于洛必达法则是必须的:
  \begin{enumerate}
    \item 形式的不正确(Indeterminacy of form): $\lim_{x\to c} f(x) = \lim_{x\to c} g(x) = 0$ 或 $\pm\infty$;以及
    \item 函数的可导性(Differentiability of functions):$f(x)$ 与 $g(x)$ 在开区间 $\mathcal{I}$ 是可微的,除了其中的点 $c$;以及
    \item 分母导数非零:$g'(x) \ne 0$ 对于所有在 $\mathcal{I}$ 的 $x$ 且 $x \ne c$;以及
    \item 导数之商存在极限:$\lim_{x\to c} \frac{f'(x)}{g'(x)}$ 存在。
  \end{enumerate}

  上述任意一个条件不成立,那么洛必达法则通常而言就不成立。
\end{anote}

\end{document}
