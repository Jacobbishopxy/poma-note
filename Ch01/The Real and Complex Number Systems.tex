\documentclass[../poma-notes.tex]{subfiles}

\begin{document}

\subsection*{Ordered Sets}

\setcounter{poma}{6}
\begin{definition}
	Suppose $S$ is an ordered set, $E \subset S$, and $E$ is bounded above.
	Suppose there exists an $\alpha \in S$ with the following properties:
	\begin{enumerate}[label=(\roman*)]
		\item $\alpha$ is an upper bound of $E$.
		\item If $\gamma < \alpha$ the $\gamma$ is not an upper bound of $E$.
	\end{enumerate}

	Then $\alpha$ is called the \textit{least upper bound of E} or the \textit{supremum of E},
	and we write
	\[ \alpha = \sup E \]
	The \textit{greatest lower bound}, or \textit{infimum}, of a set $E$ which is bounded below
	is defined in the same manner: The statement \[ \alpha = \inf E \] means that $\alpha$ is
	a lower bound of $E$ and that no $\beta$ with $\beta > \alpha$ is a lower bound of $E$.
\end{definition}

\anote
$S$ 是有序集合的情况下,$E$ 又是属于 $S$ 的,并且 $E$ 拥有上界。那么只会存在一个 $\alpha$ 是 $E$ 的最小上界。
同理如果是 $E$ 拥有下界,只会存在一个 $\alpha$ 是 $E$ 的最大下界。
发音:Supremum [su:$'$pri:məm];Infimum [$'$\i nfa\i məm]。

\setcounter{poma}{9}
\begin{definition}
	An ordered set $S$ is said to have the \textit{least-upper-bound property} if the following
	is true: If $E \subset S$, $E$ is not empty, and $E$ is bounded above, then $\sup E$ exists in $S$.
\end{definition}

\anote
$S$ 中存在 $E \subset S$,且 $E$ 具有最小上界,那么 $S$ 就具有最小上界性,反之亦然。

\begin{theorem}
	Suppose $S$ is an ordered set with the least-upper-bound property,
	$B \subset S$, $B$ is not empty, and $B$ is bounded below.
	Let $L$ be the set of all lower bounds of $B$. Then $\alpha = \sup L$
	exists in $S$, and $\alpha = \inf B$.
	In particular, $\inf B$ exists in $S$.
\end{theorem}

\begin{proof}
	因为 $B$ 是有下界的,且 $L$ 不为空。由于 $L$ 包含了所有的 $y$($y \in S$)且满足不等式 $y \leq x$($x \in B$),
	那么所有的 $x \in B$ 都是 $L$ 的上界。因此 $L$ 是有上界的。关于 $S$ 的假设意为在 $S$ 中有一个 $L$ 的最小上界,
	被称为 $\alpha$。

	如果 $\gamma < \alpha$ 那么(根据 Definition 1.8)$\gamma$ 并不是 $L$ 的一个上界,因此 $\gamma \notin B$。
	对于所有的 $x \in B$ 都有 $\alpha \le x$。因此 $\alpha \in L$。

	如果 $\alpha < \beta$ 那么 $\beta \notin L$,因为 $\alpha$ 是 $L$ 的一个上界。

	我们展示过了 $\alpha \in L$ 但是 $\beta \notin L$ 而 $\beta > \alpha$ 的情况。也就是说,$\alpha$
	是 $B$ 的一个下界,但是当 $\beta > \alpha$ 时 $\beta$ 却不是。这就意味着 $\alpha = \inf B$。
\end{proof}

\subsection*{Fields}

\begin{definition}
	A field is a set $F$ with two operations, called \textit{addition} and \textit{multiplication},
	which satisfy the following so-called "field axioms" (A), (M), and (D):

	\begin{itemize}
		\item[] \textbf{(A) Axioms for addition}
			\begin{itemize}
				\item[] (A1) If $x \in F$ and $y \in F$, then their sum $x + y$ is in $F$.
				\item[] (A2) Addition is commutative: $x + y = y + x$ for all $x,y \in F$.
				\item[] (A3) Addition is associative: $(x+y)+z=x+(y+z)$ for all $x,y,z \in F$.
				\item[] (A4) $F$ contains an element 0 such that $0+x=x$ for every $x \in F$.
				\item[] (A5) To every $x \in F$ corresponds an element $-x \in F$ such that $x+(-x)=0$.
			\end{itemize}
		\item[] \textbf{(M) Axioms for multiplication}
			\begin{itemize}
				\item[] (M1) If $x \in F$ and $y \in F$, then their product $xy$ is in $F$.
				\item[] (M2) Multiplication is commutative: $xy = yx$ for all $x,y \in F$.
				\item[] (M3) Multiplication is associative: $(xy)z = x(yz)$ for all $x,y,z \in F$.
				\item[] (M4) $F$ contains an element $1 \ne 0$ such that $1x = x$ for every $x \in F$.
				\item[] (M5) If $x \in F$ and $x \ne 0$ then there exists an element $\frac{1}{x} \in F$
					such that $x \cdot (\frac{1}{x}) = 1$.
			\end{itemize}
		\item[] \textbf{(D) The distributive law}
			\begin{itemize}
				\item[] $x(y+z) = xy+xz$ holds for all $x,y,z \in F$.
			\end{itemize}
	\end{itemize}
\end{definition}

\anote
域的定义:\href{https://en.wikipedia.org/wiki/Field_(mathematics)}{维基百科}。

\setcounter{poma}{16}
\begin{definition}
	An \textit{ordered field} is a field $F$ which is also an ordered set, such that:

	\begin{enumerate}
		\item $x+y<x+z$ if $x,y,z \in F$ and $y<z$,
		\item $xy>0$ if $x \in F$, $y \in F$, $x>0$, and $y>0$.
	\end{enumerate}
\end{definition}

如果 $x>0$,我们称 $x$ 为 \textit{positive};如果 $x<0$,$x$ 则为 \textit{negative}。

\subsection*{The Real Field}

\setcounter{poma}{18}
\begin{theorem}
	There exists an ordered field $R$ which has the least-upper-bound property.
	Moreover, $R$ contains $Q$ as a subfield.
\end{theorem}

第二个声明意味着 $Q \subset R$ 以及加法与乘法在 $R$ 上的运算,当应用于 $Q$ 的成员时,与有理数的通常操作重合;
同样的,正有理数成员是 $R$ 的正元素。

R 的成员被称为 \textit{real numbers},即实数。

% TODO
% \begin{proof}
% \end{proof}

\begin{theorem}
	\leavevmode
	\begin{enumerate}[label=(\alph*)]
		\item If $x \in R$, $y \in R$, and $x>0$, then there is a positive integer $n$ such that $nx>y$.
		\item If $x \in R$, $y \in R$, and $x<y$, then there exists a $p \in Q$ such that $x<p<y$.
	\end{enumerate}
\end{theorem}

对于 (a) 部分通常认为是 $R$ 具有 \textit{archimedean property},即阿基米德性质,
详见\href{https://en.wikipedia.org/wiki/Archimedean_property}{维基百科}。
(b) 部分则表明 $Q$ 是在 $R$ 中 \textit{dense},即具有稠密性:在任意两个实数之间有一个有理数。

\begin{proof}
	\begin{enumerate}[label=(\alph*)]
		\item 令 $A$ 作为所有 $nx$ 的集合,其中 $n$ 为所有的正整数。如果 (a) 是错误的,那么 $y$ 则会是 $A$ 的一个上界。
		      但是接着 $A$ 会在 $R$ 中拥有一个最小上界,即 $\alpha  = \sup A$。由于 $x>0$,$\alpha - x < \alpha$,
		      以及 $\alpha - x$ 不是 $A$ 的上界,因此 $\alpha - x < mx$ 对于某些正整数 $m$ 成立。但是这样就会有
		      $\alpha < (m+1)x \in A$,这是不可能的,因为 $\alpha$ 是 $A$ 的上界。
		\item 因为 $x<y$,$y-x>0$ 以及由 (a) 所知一个正整数 $n$ 满足
		      \[n(y-x)>1\]
		      再次应用 (a),获取正整数 $m_1$ 与 $m_2$ 满足 $m_1 > nx$,$m_2 > -nx$,那么
		      \[-m_2 < nx < m_1\]
		      因此会有一个整数 m( $-m_2 \le m \le m_1$)满足
		      \[m-1 \le nx \le m\]
		      如果我们结合这些不等式,则会得到
		      \[nx < m \le 1 + nx < ny\]
		      因为 $n>0$,它遵循
		      \[x < \frac{m}{n} < y\]
		      通过 $p=\frac{m}{n}$ 证明了 (b)。
	\end{enumerate}
\end{proof}

\anote
(b) 的第一步将 $y-x$ 作为整体,将 (a) $nx > y$ 中的 $x$ 替换为 $y-x$,$y$ 替换为 1。同理对于整数
$m_1$ 与 $m_2$ 而言,可分别将 $nx$ 与 $-nx$ 视为不等式右侧的 $y$,而不等式左侧的 $x$ 视为 1,
那么就有了 $m_1 \cdot 1 > nx$ 与 $m_2 \cdot 1 > -nx$。对于整数 $m$ 的 $-m_2 \le m \le m_1$ 最坏的情况
可将 $-m_2$ 与 $m_1$ 视为相邻的整数,比如说 1 和 2,那么当 $m$ 取值为 2 时视为 $2 - 1 \le nx < 2$,
满足 $m - 1 \le nx < m$。最后根据有理数的定义 $p=m/n$ $m,n \in Q$ 可以得出 $x$ 与 $y$ 之间一定存在一个有理数。

\begin{theorem}
	For every real $x>0$ and every integer $n>0$ there is one and only one positive real $y$ such that
	$y^n=x$.
\end{theorem}

这个 $y$ 数可以被写作 $\sqrt[n]{x}$ 或是 $x^\frac{1}{n}$。

\begin{proof}
	对于至多存在一个 $y$ 的论证很简单,因为 $0 < y_1 < y_2$ 意味着 $y^n_1 < y^n_2$。

	令 $E$ 为所有满足 $t^n < x$ 的正实数 $t$ 的集合。

	如果 $t = x/(1+x)$ 那么 $0 \le t < 1$,那么 $t^n \le t < x$。因此 $t \in E$,且 $E$ 不为空。

	如果 $t>1+x$ 那么 $t^n \ge t > x$,所以 $t \notin E$。因此 $1+x$ 是 $E$ 的一个上界。

	所以根据 Theorem 1.19 得出,存在一个
	\[ y = \sup E\]

	而证明 $y^n = x$ 我们需要展示不等式 $y^n < x$ 与 $y^n > x$ 皆会导致矛盾。

	当  $0<a<b$ 时,等式 $b^n - a^n = (b-a)(b^{n-1} + b^{n-2}a + \dots + a^{n-1})$ 得出不等式
	\[b^n - a^n < (b-a)nb^{n-1}\]

	假设 $y^n<x$。选择 h 使得 $0<h<1$ 且
	\[h < \frac{x-y^n}{n(y+1)^{n-1}}\]

	令 $a=y$, $b=y+h$,那么就有
	\[(y+h)^n-y^n < hn(y+h)^{n-1} < hn(y+1)^{n-1} < x-y^n\]

	因此 $(y+h)^n < x$,且 $y+h \in E$。因为 $y+h>y$,这与 $y$ 是 $E$ 的一个上界相矛盾。

	假设 $y^n>x$。令
	\[k=\frac{y^n-x}{ny^{n-1}}\]

	那么 $0<k<y$。如果 $t \ge y-k$,我们得出以下结论:
	\[y^n - t^n \le y^n - (y - k)^n \le kny^{n-1} = y^n - x\]

	因此 $t^n > x$,且 $t \notin E$。它遵循 $y - k$ 是 $E$ 的一个上界。

	但是因为 $y-k<y$,其与 $y$ 是 $E$ 的\textit{最小}上界的事实相矛盾。

	因此 $y^n=x$,证明完成。
\end{proof}

\begin{anote}
	至多存在一个 $y$ 换个角度也就是说但凡有第二个 $y$ 使得 $y^n_1 = y^n_2$,那么 $y_1 = y_2$。\\
	等式
	\[b^n - a^n = (b-a)(b^{n-1} + b^{n-2}a + \dots + a^{n-1})\]
	左侧可以视为
	\[(b-a)(b^{n-1} + b^{n-1}\frac{a}{b} + \dots + b^{n-1}\frac{a}{b^{n-1}})\]
	提取 $b^{n-1}$ 后得
	\[(b-a)b^{n-1}(1 + \frac{a}{b} + \dots + \frac{a^{n-1}}{b^{n-1}})\]
	由于有 $0<a<b$ 这么一个前提,可以将第三项变为
	\[1 + \frac{a}{b} + \dots + \frac{a^{n-1}}{b^{n-1}} < 1 + 1 + \dots + 1 = n\]
	因此可得不等式
	\[b^n - a^n < (b-a)nb^{n-1}\]
	至于在证明 $y^n < x$ 不成立时,选择 $h$ 的 $h<\frac{x-y^n}{n(y+1)^{n-1}}$ 分母为什么是 $n(y+1)^{n-1}$,
	是因为这是为了之后处理不等式而特意设置的消除项(这里利用了函数 $f(x)=x^n$ 是连续的事实,也就是说分母一定也是实数,
	那么就可以将 $h$ 视为小于某实数);同样的在证明 $y^n > x$ 时的 $k$ 也是如此。
\end{anote}

\begin{corollary}
	If $a$ and $b$ are positive real numbers and $n$ is a positive integer, then
	\[(ab)^{1/n} = a^{1/n}b^{1/n}\]
\end{corollary}

\begin{proof}
	令 $\alpha = a^{1/n}$,$\beta = b^{1/n}$,那么有
	\[ab = \alpha^n \beta^n = (\alpha\beta)^n\]
	而乘法是符合交换律的,因此
	\[(ab)^{1/n} = \alpha\beta = a^{1/n}b^{1/n}\]
\end{proof}

\subsection*{The Extended Real Number System}

\setcounter{poma}{22}
\begin{definition}
	The extended real number system consists of the real field $R$ and two symbols, $+\infty$ and $-\infty$.
	We preserve the original order in $R$, and define
	\[-\infty < x < +\infty\]
	for every $x \in R$.
\end{definition}

可以清楚的知道 $+\infty$ 是所有衍生的实数系统子集的一个上界,且每个非空子集都有一个最小上界。
如果 $E$ 是一个实数的非空集合,且没有上界在 $R$ 中,那么 $\sup E = + \infty$ 在衍生实数系统中。

下界同理。

衍生实数系统并不形成一个域,但它形成了一下惯例:

\begin{enumerate}[label=(\alph*)]
	\item 如果 $x$ 是实数则
	      \[x+\infty=+\infty,\quad x-\infty=-\infty,\quad \frac{x}{+\infty}=\frac{x}{-\infty}=0\]
	\item 如果 $x>0$ 则 $x \cdot (+\infty) = +\infty,\ x \cdot (-\infty) = -\infty$
	\item 如果 $x<0$ 则 $x \cdot (+\infty) = -\infty,\ x \cdot (-\infty) = +\infty$
\end{enumerate}

\subsection*{The Complex Field}

\begin{definition}
	A \textit{complex number} is an ordered pair $(a, b)$ of real numbers. "Ordered" means that $(a, b)$ and $(b, a)$
	are regarded as distinct if $a \neq b$.
\end{definition}

令 $x=(a,b)$,$y=(c,d)$ 为两个复数。当且仅当 $a=c$ 以及 $b=d$ 时有 $x=y$。(注意该定义并非是完全不必要的;
考虑有理数的等式,表现为整数的商。)我们定义:
\[x+y=(a+c,b+d)\]
\[xy=(ac-bd,ad+bc)\]

\begin{anote}
	作为补充(详见\href{https://zhuanlan.zhihu.com/p/68763358}{该篇文章}),
	对于任意两个复数 $x=(a,b),\ y=(c,d)$ 的四则运算:
	\begin{enumerate}[label=(\arabic*)]
		\item $x+y=(a+c)+i(b+d)$
		\item $x-y=(a-c)+i(b-d)$
		\item $xy=(a+ib)(c+id)=ac-bd+i(ad+bc)$
		\item $\frac{x}{y} = \frac{(a+ib)(c-id)}{(c+id)(c-id)} = \frac{(ac+bd)+i(bc-ad)}{c^2+d^2}$
	\end{enumerate}

	首先,对于任意一个复数 $z=a+ib$,可以用复平面上的一个点来表示,那么复数加减法可以通过向量来理解:
	\begin{center}
		\begin{tikzpicture}[line cap=round,line join=round,>=triangle 45,x=0.8cm,y=0.8cm]
			\begin{axis}[
					x=0.8cm,y=0.8cm,
					axis lines=middle,
					ymajorgrids=true,
					xmajorgrids=true,
					xlabel=Re,
					ylabel=Im,
					xmin=-0.5,
					xmax=10.5,
					ymin=-0.5,
					ymax=10.5,
					xtick={0,1,...,10},
					ytick={0,1,...,10},]
				\draw [->,line width=1.5pt] (0,0) -- (2,5);
				\draw [->,line width=1.5pt] (0,0) -- (6,2);
				\draw [->,line width=1.5pt,dash pattern=on 10pt off 10pt] (2,5) -- (7,7);
				\draw [->,line width=1.5pt,color=red] (0,0) -- (7,7);
				\begin{scriptsize}
					\draw [fill=black] (0,0) circle (2pt);
					\draw [fill=black] (2,5) circle (2.5pt);
					\draw[color=black] (2.2,5.6) node {\large A};
					\draw [fill=black] (6,2) circle (2.5pt);
					\draw[color=black] (6.2,2.6) node {\large B};
					\draw [fill=black] (7,7) circle (2.5pt);
					\draw[color=black] (7.2,7.6) node {\large C};
				\end{scriptsize}
			\end{axis}
		\end{tikzpicture}
	\end{center}
	\begin{center}
		\begin{tikzpicture}[line cap=round,line join=round,>=triangle 45,x=0.8cm,y=0.8cm]
			\begin{axis}[
					x=0.8cm,y=0.8cm,
					axis lines=middle,
					ymajorgrids=true,
					xmajorgrids=true,
					xlabel=Re,
					ylabel=Im,
					xmin=-0.5,
					xmax=10.5,
					ymin=-0.5,
					ymax=10.5,
					xtick={0,1,...,10},
					ytick={0,1,...,10},]
				\draw [->,line width=1.5pt] (0,0) -- (2,8);
				\draw [->,line width=1.5pt,color=brown] (0,0) -- (8,6);
				\draw [->,line width=1.5pt,color=red] (8,6) -- (2,8);
				\draw (4,2.7) node[anchor=north west] {Reverse B's direction};
				\begin{scriptsize}
					\draw [fill=black] (0,0) circle (2pt);
					\draw[color=black] (2.2,8.7) node {\large A};
					\draw[color=black] (8.2,6.7) node {\large B};
				\end{scriptsize}
			\end{axis}
		\end{tikzpicture}
	\end{center}

	而对于复数的乘除法,需要先引入复数在极坐标上的几何意义 -- 对于任意一个复数 $z=a+ib$ 用极坐标来表示:
	\begin{center}
		\begin{tikzpicture}[line cap=round,line join=round,>=triangle 45,x=0.8cm,y=0.8cm]
			\begin{axis}[
					x=0.8cm,y=0.8cm,
					axis lines=middle,
					ymajorgrids=true,
					xmajorgrids=true,
					xlabel=Re,
					ylabel=Im,
					xmin=-0.5,
					xmax=10.5,
					ymin=-0.5,
					ymax=10.5,
					xtick={0,1,...,10},
					ytick={0,1,...,10},]
				\draw [line width=1.5pt,fill opacity=0.1] (0,0) -- (0:3) arc (0:26.565:3) -- cycle;
				\draw (0.3,6) node[anchor=north west] {Imaginary part as YAxis};
				\draw (4,0.7) node[anchor=north west] {Real part as XAxis};
				\draw [line width=1.5pt] (0,0)-- (8,4);
				\draw (8.7,4.45) node[anchor=north west] {\large(x,y)};
				\begin{scriptsize}
					\draw [fill=black] (0,0) circle (3pt);
					\draw[color=black] (0.35,0.55) node {\large O};
					\draw [fill=black] (8,4) circle (2.5pt);
					\draw[color=black] (8.5,4) node {\large P};
					\draw[color=black] (3.2,1) node {\large $\theta$};
				\end{scriptsize}
			\end{axis}
		\end{tikzpicture}
	\end{center}
	那么复数的三角表示为:
	\begin{center}
		\begin{tikzpicture}[line cap=round,line join=round,>=triangle 45,x=0.8cm,y=0.8cm]
			\begin{axis}[
					x=0.8cm,y=0.8cm,
					axis lines=middle,
					ymajorgrids=true,
					xmajorgrids=true,
					xlabel=Re,
					ylabel=Im,
					xmin=-0.5,
					xmax=10.5,
					ymin=-0.5,
					ymax=10.5,
					xtick={0,1,...,10},
					ytick={0,1,...,10},]
				\draw [line width=1.5pt,fill opacity=0.1] (0,0) -- (0:3) arc (0:26.565:3) -- cycle;
				\draw [line width=1.5pt] (0,0)-- (8,4);
				\draw (8.7,4.45) node[anchor=north west] {\large(x,y)};
				\draw [line width=1pt ] (8,0)-- (8,4);
				\draw [line width=1pt ] (7.5,0)-- (7.5,0.5);
				\draw [line width=1pt ] (7.5,0.5)-- (8,0.5);
				\draw (8.1,2) node[anchor=north west] {\large y};
				\draw (5,0.6) node[anchor=north west] {\large x};
				\begin{scriptsize}
					\draw [fill=black] (0,0) circle (3pt);
					\draw[color=black] (0.35,0.55) node {\large O};
					\draw [fill=black] (8,4) circle (2.5pt);
					\draw[color=black] (8.5,4) node {\large P};
					\draw[color=black] (3.2,1) node {\large $\theta$};
				\end{scriptsize}
			\end{axis}
		\end{tikzpicture}
	\end{center}
	这里将 $OP$ 的长度作为复数 $z$ 的\textbf{模(Modulus)},用 $|z|$ 表示;而角 $\theta$ 为复数 $z$ 的\textbf{幅角(Argument)},
	用 $\arg(z)$ 表示。那么复数的三角表示为:
	\[z = x+iy = r(\cos\theta + i\sin\theta), \quad r = \sqrt{x^2+y^2}, \quad \tan\theta = \frac{y}{x} \]
	接下来是复数乘法的几何意义,使用复数的三角形式计算下列两个复数的乘积:
	\begin{align*}
		z_1 = r_1(\cos\theta_1 + i\sin\theta_1) \\
		z_2 = r_2(\cos\theta_2 + i\sin\theta_2)
	\end{align*}
	那么有:
	\begin{align*}
		\mathcal{} z_1 z_2 & = r_1 r_2(\cos\theta_1 + i\sin\theta_1)(\cos\theta_2 + i\sin\theta_2)                                                                  \\
		                   & = r_1 r_2 \big( \cos\theta_1 \cos\theta_2 - \sin\theta_1 \sin\theta_2 + i(\sin\theta_1 \cos\theta_2 + \cos\theta_1 \sin\theta_2) \big)
	\end{align*}
	那么根据三角和差公式:
	\[z_1 z_2 = r_1 r_2 \big( \cos(\theta_1 + \theta_2) + i\sin(\theta_1 + \theta_2) \big)\]
	可以发现 $z_1 \cdot z_2$ 计算后模为两个复数模的乘积 $|z_1||z_2|$,幅角为两个复数幅角之和 $\arg(z_1z_2)=\arg(z_1)+\arg(z_2)$。
	因此复数的乘积可以理解为\textbf{拉伸与旋转}。例如:
	\begin{center}
		\begin{tikzpicture}[line cap=round,line join=round,>=triangle 45,x=0.8cm,y=0.8cm]
			\begin{axis}[
					x=0.8cm,y=0.8cm,
					axis lines=middle,
					ymajorgrids=true,
					xmajorgrids=true,
					xmin=-7.5,
					xmax=7.5,
					ymin=-1.5,
					ymax=7.5,
					xtick={-8,-7,...,7},
					ytick={-1,0,...,8},]
				\draw [line width=2pt] (0,0)-- (6,6);
				\draw [line width=2pt,color=red] (0,0)-- (-6,6);
				\draw [->,line width=2pt,dash pattern=on 10pt off 10pt,color=blue] (6,6) -- (-6,6);
				\draw (3.5,3.7) node[anchor=north west] {\large $z_1 = 1 + i$};
				\draw [line width=2pt] (0,0)-- (0,6);
				\draw (0.1,4) node[anchor=north west] {\large $z_2 = i$};
				\draw (-6.8,4.7) node[anchor=north west] {\large $z_1z_2 = (1+i)\cdot i = -1 + i$};
				\begin{scriptsize}
					\draw [fill=black] (0,0) circle (2pt);
					\draw [fill=black] (6,6) circle (2.5pt);
					\draw [fill=red] (-6,6) circle (2.5pt);
					\draw [fill=black] (0,6) circle (2.5pt);
				\end{scriptsize}
			\end{axis}
		\end{tikzpicture}
	\end{center}
	因为
	\[z_1=\sqrt{2}(\cos\frac{\pi}{4}+i\sin\frac{\pi}{4}),\quad z_2=1(\cos\frac{\pi}{2}+i\sin\frac{\pi}{2})\]
	所以 $z_1z_2$ 的模长为 $\sqrt{2}$ 且幅角为 $\frac{3\pi}{4}$。
	而复数的除法只需要将除法写成乘法形式即可
	\begin{align*}
		\mathcal{} z^{-1} & = \big(r(\cos\theta+i\sin\theta)\big)^{-1}                                             \\
		                  & = r^{-1}\frac{1}{\cos\theta+i\sin\theta}                                               \\
		                  & =r^{-1}\frac{\cos\theta-i\sin\theta}{(\cos\theta+i\sin\theta)(\cos\theta-i\sin\theta)} \\
		                  & =r^{-1}(\cos\theta-i\sin\theta)
	\end{align*}
	那么两个复数
	\begin{align*}
		z_1 = r_1(\cos\theta_1 + i\sin\theta_1) \\
		z_2 = r_2(\cos\theta_2 + i\sin\theta_2)
	\end{align*}
	的除法便是
	\begin{align*}
		\mathcal{} \frac{z_1}{z_2} & = z_1{z_2}^{-1}                                                           \\
		                           & = \frac{r_1}{r_2}(\cos\theta_1+i\sin\theta_2)(\cos\theta_1-i\sin\theta_2) \\
		                           & = \frac{r_1}{r_2}(\cos(\theta_1-\theta_2)+i\sin(\theta_1-\theta_2))
	\end{align*}
	因此,$\frac{z_1}{z_2}$ 的模长为两个模相除 $\frac{|z_1|}{|z_2|}$,幅角为 $\arg(\frac{z_1}{z_2})=\arg(z_1)-\arg(z_2)$。
	所以复数的除法也可以理解为\textbf{拉伸与旋转}。

	综上所述,复数的加减法就是向量的加减法,乘除法就是拉伸与旋转变换。
\end{anote}

\setcounter{poma}{24}
\begin{theorem}
	These definitions of addition and multiplication turn the set of all complex numbers into a field,
	with (0, 0) and (1, 0) in the role of 0 and 1.
\end{theorem}

\begin{proof}
	我们简单的验证一下域的公理(Definition 1.12),使用 $R$ 的域结构。
	令 $x=(a,b), y=(c,d), z=(e,f)$。
	\begin{itemize}
		\item[] (A1) 很清楚。
		\item[] (A2) $x+y = (a+c,b+d) = (c+a,d+b) = y+x$
		\item[] (A3)
			\vspace{-26pt}
			\begin{align*}
				\mathcal (x+y)+z & = (a+c,b+d) + (e,f) \\
				                 & = (a+c+e, b+d+f)    \\
				                 & = (a,b) + (c+e,d+f) \\
				                 & = x+(y+z)
			\end{align*}
		\item[] (A4) $x+0 = (a,b) + (0,0) = (a,b) = x$
		\item[] (A5) 令 $-x = (-a,-b)$,那么 $x+(-x) = (0,0) = 0$
		\item[] (M1) 很清楚。
		\item[] (M2) $xy = (ac-bd,ad+bc) = (ca-db,da+cb) = yx$
		\item[] (M3)
			\vspace{-26pt}
			\begin{align*}
				\mathcal (xy)z & = (ac-db,ad+bc)(e,f)                \\
				               & = (ace-bde-adf-bef,acf-bdf+ade+bce) \\
				               & = (a,b)(ce-df,cf+de)                \\
				               & = x(yz)
			\end{align*}
		\item[] (M4) $1x  =(1,0)(a,b) = (a,b) = x$
		\item[] (M5) 如果 $x \ne 0$ 那么 $(a,b) \ne (0,0)$,也就是说 $a$ 和 $b$ 至少有一个实数不等于 0。
			因此 $a^2+b^2>0$,根据 Proposition 1.18(d),我们可以定义
			\[\frac{1}{x} = \bigl(\frac{a}{a^2+b^2}, \frac{-b}{a^2+b^2}\bigr)\]
			那么
			\[x \cdot \frac{1}{x} = (a,b)\bigl(\frac{a}{a^2+b^2}, \frac{-b}{a^2+b^2}\bigr) = (1,0) = 1\]
		\item[] (D)
			\vspace{-26pt}
			\begin{align*}
				\mathcal{} x(y+z) & = (a,b)(c+e,d+f)                \\
				                  & = (ac+ae-bd-bf,ad+af+bc+be)     \\
				                  & = (ac-bd,ad+bc) + (ae-bf,af+be) \\
				                  & = xy + yz
			\end{align*}
	\end{itemize}
\end{proof}

\begin{theorem}
	For any real numbers $a$ and $b$ we have
	\[(a,0) + (b,0) = (a+b,0), \quad (a,0)(b,0) = (ab,0)\]
\end{theorem}

\begin{definition}
	$i=(0,1)$
\end{definition}

\setcounter{poma}{27}
\begin{theorem}
	$i^2=-1$
\end{theorem}

\begin{proof}
	$i^2 = (0,1)(0,1) = (-1,0) = -1$
\end{proof}

\begin{theorem}
	If a and b are real, then $(a,b)=a+bi$
\end{theorem}

\begin{proof}
	\vspace{-26pt}
	\begin{align*}
		\mathcal{} a+bi & = (a,0) + (b,0)(0,1) \\
		                & = (a,0) + (0,b)      \\
		                & = (a,b)
	\end{align*}
\end{proof}

\begin{definition}
	If $a$, $b$ are real and $z=a+bi$, then the complex number $\overline{z}=a-bi$ is called the \textit{conjugate} of $z$.
	The numbers $a$ and $b$ are the \textit{real part} and the \textit{imaginary part} of $z$, respectively.
	We shall occasionally write
	\[ a = Re(z),\quad b=Im(z)\]
\end{definition}

\begin{theorem}
	If $z$ and $w$ are complex, then
	\begin{enumerate}[label=(\alph*)]
		\item $\overline{z + w} = \overline{z} + \overline{w}$
		\item $\overline{zw} = \overline{z} \cdot \overline{w}$
		\item $z + \overline{z} = 2 \, Re(z), \, z - \overline{z} = 2i \, Im(z)$
		\item $z\overline{z}$ is real and positive (except when $z=0$)
	\end{enumerate}
\end{theorem}

\begin{definition}
	If $z$ is a complex number, its absolute value $|z|$ is the non-negative square root of $z\overline{z}$;
	that is, $|z| = (z\overline{z})^{1/2}$.
\end{definition}

$|z|$ 的存在(以及唯一性)遵循 Theorem 1.12 以及 Theorem 1.31 (d)。

注意当 $x$ 为实数时,那么 $\overline{x} = x$,因此 $|x| = \sqrt{x^2}$。
所以如果 $x \ge 0$ 时 $|x| = x$,如果 $x<0$ 时 $|x| = -x$。

\begin{theorem}
	Let $z$ and $w$ be complex numbers. Then
	\begin{enumerate}[label=(\alph*)]
		\item $|z|>0$ unless $z=0,\,|0|=0$
		\item $|\overline{z}|=|z|$
		\item $|zw|=|z||w|$
		\item $|Re \, z| \le |z|$
		\item $|z+w| \le |z|+|w|$
	\end{enumerate}
\end{theorem}

\begin{proof}
	(a) 与 (b) 不足为道。令 $z=a+bi,\,w=c+di$,其 $a,\,b,\,c,\,d$ 皆为实数。那么
	\[|zw|^2 = (ac-bd)^2 + (ad+bc)^2 = (a^2+b^2)(c^2+d^2) = |z|^2|w|^2\]
	即 $|zw|^2 = (|z||w|)^2$。(c) 遵循 Theorem 1.21 所声明的唯一性。

	证明 (d),有 $a^2 \le a^2 + b^2$,因此
	\[|a| = \sqrt{a^2} \le \sqrt{a^2 + b^2}\]

	证明 (e),有 $\overline{z}w$ 与 $z\overline{w}$ 是共轭的,因此 $\overline{z}w + z\overline{w} = 2\, Re\,(z\overline{w})$。
	因此
	\begin{align*}
		\mathcal |z+w|^2 & = (z+w)(\overline{z}+\overline{w})                              \\
		                 & = z\overline{z} + z\overline{w} + \overline{z}w + w\overline{w} \\
		                 & = |z|^2 + 2\,Re\,(z\overline{w}) + |w|^2                        \\
		                 & \le |z|^2 + 2|z\overline{w}| + |w|^2                            \\
		                 & = |z|^2 + 2|z||w| + |w|^2                                       \\
		                 & = (|z|+|w|)^2
	\end{align*}

	两边开根号后即可得 (e)。
\end{proof}

\anote
计算中第一步的 $|z+w|^2 = (z+w)(\overline{z}+\overline{w})$ 是将 $z+w$ 视为整体,并使用了 Definition 1.32 中的
$|z|=(z\overline{z})^{1/2}$ 转换为 $(z+w)(\overline{z+w})$,而又因为 Theorem 1.31 (a) 可将第二项变为
$(\overline{z}+\overline{w})$;而第三步到第四步的不等式则利用了 Theorem 1.33 (d),即 $|Re z| \le |z|$;
第四步到第五步则使用了 Theorem 1.33 (c),即 $|zw|=|z||w|$。

\begin{notation}
	If $x_1,\dots,x_n$ are complex numbers, we write
	\[x_1 + x_2 + \cdots + x_n = \sum_{j=1}^{n} x_j\]
\end{notation}

我们用一个重要的不等式来结束本节,它通常被称为 \textit{Schwarz inequality}。

\begin{theorem}
	If $a_1,\dots,a_n$ and $b_1,\dots,b_n$ are complex numbers, then
	\[\left|\sum_{j=1}^{n} a_j \overline{b}_j\right|^2 \le \sum_{j=1}^{n}|a_j|^2\sum_{j=1}^{n}|b_j|^2\]
\end{theorem}

\begin{proof}
	令 $A=\sum|a_j|^2, \, B=\sum|b_j|^2, \, C=\sum a_j\overline{b}_j$,本证明中 j 取值 $1,\dots,n$。
	如果 $B=0$,那么就有 $b_1=\cdots=b_n=0$,那么结论很清楚。因此假设 $B>0$。根据 Theorem 1.31 有
	\begin{align*}
		\mathcal{} \sum|Ba_j - Cb_j|^2 & = \sum (Ba_j - Cb_j)(B\overline{a}_j - \overline{CB_j})                                              \\
		                               & = B^2\sum|a_j|^2 - B\overline{C}\sum a_j\overline{b}_j - BC\sum\overline{a}_j b_j + |C|^2\sum|b_j|^2 \\
		                               & = B^2A - B|C|^2                                                                                      \\
		                               & = B(AB - |C|^2)
	\end{align*}
	因为在首次求和的每个项都是非负的,可以知道
	\[B(AB-|C|^2)\ge0\]
	又因为 $B>0$,它遵循 $AB-|C|^2\ge0$。它便是预期的不等式。
\end{proof}

\anote
$|Ba_j - Cb_j|^2$ 为构造项,将其中 $Ba_j - Cb_j$ 视为整体根据 Definition 1.32,可转换为 $(Ba_j - Cb_j)(\overline{Ba_j - Cb_j})$,
而后者根据 Theorem 1.31 即可写为 $B\overline{a}_j - \overline{Cb_j}$(这里 $B=\sum|b_j|^2$ 的共轭还是其本身);
根据乘法分配律得出第二步后,将之前设定的 $A, B, C$ 带入即可得出 $B(AB - |C|^2)$;最后根据起始的构造项 $|Ba_j - Cb_j|^2$
必然非负以及之前假设的 $B>0$ 可以得出 $AB-|C|^2 \ge 0$;将 $A, B, C$ 原本代表的值带入,
即 $\sum|a_j|^2\sum|b_j|^2-|\sum a_j\overline{b}_j|\ge0$。

\subsection*{Euclidean Spaces}

\begin{definition}
	For each positive integer $k$, let $R^k$ be the set of all ordered $k$-tuples
	\[\mathbf{x}=(x_1,x_2,\dots,x_k)\]
	where $x_1,\dots,x_k$ are real numbers, called the \textit{coordintates} of $\mathbf{x}$. The elements of $R^k$
	are called points, or vectors, especially when $k>1$. We shall denote vectors by boldfaced letters.
	If $\mathbf{y}=(y_1,\dots,y_k)$ and if $\alpha$ is a real number, put
	\[\mathbf{x}+\mathbf{y}=(x_1+y_1,\dots,x_k+y_k)\]
	\[\alpha\mathbf{x}=(\alpha x_1,\dots,\alpha x_k)\]
	so that $\mathbf{x}+\mathbf{y} \in R^k$ and $\alpha\mathbf{x} \in R^k$. This defines addition of vectors,
	as well as multiplication of a vector by a real number (a scalar). These two operations satisfy the commutative,
	associative, and distributive laws (the proof is trivial, in view of the analogous for the real numbers) and
	make $R^k$ into a \textit{vector space over the real field}. The zero element of $R^k$ (sometimes called the
	\textit{origin} or the \textit{null vector}) is the point $\mathbf{0}$, all of whose coordintates are 0.

	We also define the so-called \say{inner product} (or scalar product) of $\mathbf{x}$ and $\mathbf{y}$ by
	\[\mathbf{x}\cdot\mathbf{y}=\sum_{i=1}^{k}x_i y_i\]
	and the \textit{norm} of $\mathbf{x}$ by
	\[|\mathbf{x}| = (\mathbf{x}\cdot\mathbf{x})^{1/2} = \left(\sum_{1}^{k}x^2_i\right)^{1/2}\]

	The structure now defined (the vector space $R^k$ with the above inner product and norm) is called
	euclidean $k$-space.
\end{definition}

\begin{theorem}
	Suppose $\mathbf{x},\mathbf{y},\mathbf{z}\in R^k$, and $\alpha$ is real. Then
	\begin{enumerate}[label=(\alph*)]
		\item $|\mathbf{x}| \ge 0$;
		\item $|\mathbf{x}| = 0$ if and only if $\mathbf{x} = \mathbf{0}$;
		\item $|\alpha\mathbf{x}| = |\alpha||\mathbf{x}|$;
		\item $|\mathbf{x}\cdot\mathbf{y}|\le|\mathbf{x}||\mathbf{y}|$;
		\item $|\mathbf{x}+\mathbf{y}|\le|\mathbf{x}|+|\mathbf{y}|$;
		\item $|\mathbf{x}-\mathbf{z}|\le|\mathbf{x}-\mathbf{y}|+|\mathbf{y}-\mathbf{z}|$
	\end{enumerate}
\end{theorem}

\begin{proof}
	前三项不必赘述,而 (d) 是 Schwarz 不等式的间接结论。通过 (d) 可以有
	\begin{align*}
		\mathcal{} |\mathbf{x}+\mathbf{y}|^2 & = (\mathbf{x}+\mathbf{y})\cdot(\mathbf{x}+\mathbf{y})                               \\
		                                     & =\mathbf{x}\cdot\mathbf{x} + 2\mathbf{x}\cdot\mathbf{y} + \mathbf{y}\cdot\mathbf{y} \\
		                                     & \le |\mathbf{x}|^2 + 2|\mathbf{x}||\mathbf{y}| + |\mathbf{y}|^2                     \\
		                                     & = (|\mathbf{x}|+|\mathbf{y}|)^2
	\end{align*}
	这样 (e) 就被证明了。最后替换 (e) 中的 $\mathbf{x}$ 为 $\mathbf{x}-\mathbf{y}$ 以及 $\mathbf{y}$ 为 $\mathbf{y}-\mathbf{z}$ (f)
	可以得出 (f)。
\end{proof}

\begin{remark}
	Theorem 1.37 (a), (b), and (f) will allow us (see Chap. 2) to regard $R^k$ as a metric space.

	$R^1$ (the set of all real numbers) is usually called the line, or the real line. Likewise,
	$R^2$ is called the plane, or the complex plane (compare Definitions 1.24 and 1.36).
	In these two cases the norm is just the absolute value of the corresponding real or complex number.
\end{remark}

\end{document}
