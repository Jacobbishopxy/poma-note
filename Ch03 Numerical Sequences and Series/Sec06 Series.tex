\documentclass[../poma-notes.tex]{subfiles}

\begin{document}

\subsection*{Series}

在本章的剩余部分中,所有的序列与数列都被视作复数,除了显式的声明不为复数的情况。

\begin{definition}
  Given a sequence $\{a_n\}$, we use the notation
  \[ \sum_{n=p}^{q} a_n \qquad (p \le q) \]
  to denote the sum $ a_p + a_{p+1} + \cdots + a_q$. With $\{a_n\}$ we associate a sequence $\{s_n\}$, where
  \[ s_n = \sum_{k=1}^{n} a_k. \]
  For $\{s_n\}$ we also use the symbolic expression
  \[ a_1 + a_2 + a_3 + \cdots \]
  or, more concisely,
  \begin{equation}
    \sum_{n=1}^{\infty} a_n.
  \end{equation}

  The symbol (4) we call an \textit{infinite series}, or just a \textit{series}. The numbers $s_n$ are called
  the \textit{partial sums} of the series. If $\{s_n\}$ converges to $s$, we say that the series $converges$,
  and write
  \[ \sum_{n=1}^{\infty} = s. \]
  The number $s$ is called the sum of the series; but it should be clearly understood that \textit{$s$ is the
    limit of a sequence of sums}, and is not obtained simply by addition.

  If $\{s_n\}$ diverges, the series is said to diverge.

  Sometimes, for convenience of notation, we shall consider series of the form
  \begin{equation}
    \sum_{n=0}^{\infty} a_n.
  \end{equation}
  And frequently, when there is no possible ambiguity, or when the distinction is immaterial, we shall simply
  write $\sum a_n$ in place of (4) or (5).

  It is clear that every theorem about sequences can be stated in terms of series (putting $a_1 = s_1$, and
  $a_n = s_n - s_{n-1}$ for $n > 1$), and vice versa. But it is nevertheless useful to consider both concepts.
\end{definition}

\end{document}
