\documentclass[../poma-notes.tex]{subfiles}

\begin{document}

\subsection*{Series}

在本章的剩余部分中,所有的序列与级数都被视作复数,除了显式的声明不为复数的情况。

\begin{definition}
  Given a sequence $\{a_n\}$, we use the notation
  \[ \sum_{n=p}^{q} a_n \qquad (p \le q) \]
  to denote the sum $ a_p + a_{p+1} + \cdots + a_q$. With $\{a_n\}$ we associate a sequence $\{s_n\}$, where
  \[ s_n = \sum_{k=1}^{n} a_k. \]
  For $\{s_n\}$ we also use the symbolic expression
  \[ a_1 + a_2 + a_3 + \cdots \]
  or, more concisely,
  \begin{equation}
    \sum_{n=1}^{\infty} a_n.
  \end{equation}

  The symbol (4) we call an \textit{infinite series}, or just a \textit{series}. The numbers $s_n$ are called
  the \textit{partial sums} of the series. If $\{s_n\}$ converges to $s$, we say that the series $converges$,
  and write
  \[ \sum_{n=1}^{\infty} = s. \]
  The number $s$ is called the sum of the series; but it should be clearly understood that \textit{$s$ is the
    limit of a sequence of sums}, and is not obtained simply by addition.

  If $\{s_n\}$ diverges, the series is said to diverge.

  Sometimes, for convenience of notation, we shall consider series of the form
  \begin{equation}
    \sum_{n=0}^{\infty} a_n.
  \end{equation}
  And frequently, when there is no possible ambiguity, or when the distinction is immaterial, we shall simply
  write $\sum a_n$ in place of (4) or (5).

  It is clear that every theorem about sequences can be stated in terms of series (putting $a_1 = s_1$, and
  $a_n = s_n - s_{n-1}$ for $n > 1$), and vice versa. But it is nevertheless useful to consider both concepts.
\end{definition}

\anote 级数的 \href{https://en.wikipedia.org/wiki/Series_(mathematics)}{Wiki 定义}。

\begin{anote}\mbox{}\par
  对序列 $a_n$,令 $s_n = \sum_{k=1}^{n} a_k$ 为\textbf{部分和},$\sum_{n=1}^{\infty} a_n$ 叫做无穷级数
  (\textit{infinite series}),简称级数(\textit{series})。

  如果 $s_n$ 收敛,那么级数收敛并记为 $\sum_{n=1}^{\infty} a_n = s$;如果 $s_n$ 发散,那么级数发散。
\end{anote}

\begin{theorem}
  $\Sigma a_n$ converges if and only if for every $\varepsilon > 0$ there is an integer $N$ such that
  \begin{equation}
    \Bigg| \sum_{k=n}^{m} a_k \Bigg| \le \varepsilon
  \end{equation}
  if $m \ge n \ge N$.
\end{theorem}

\begin{anote}
  柯西准则 Theorem 3.11 可以重新表述为,$\Sigma a_n$ 收敛,当且仅当,对于任意 $\varepsilon > 0$,存在级数 $N$,使得
  $m \ge n \ge N$ 时 $\Bigg| \sum_{k=n}^{m} a_k \Bigg| \le \varepsilon$。
\end{anote}

尤其是在取值 $m = n$ 的时候 (6) 则会变为
\[|a_n| \le \varepsilon \qquad (n \ge N).\]
换言之:

\begin{theorem}
  If $\Sigma a_n$ converges, then $\lim_{n \to \infty} a_n = 0$.
\end{theorem}

然而其中 $a_n \to 0$ 并不能完全确保 $\Sigma a_n$ 的收敛。例如级数

\[\sum_{n=1}^{\infty} \frac{1}{n}\]

是发散的;我们将在 Theorem 3.28 中进行证明。

Theorem 3.14 中所提出的单调序列,也有着与级数所直接对应的部分。

\anote 如果 $\Sigma a_n$ 收敛,则 $\lim_{n \to \infty} a_n = 0$。

\begin{theorem}
  A series of nonnegative terms converges if and only if its partial sums form a bounded sequence.
\end{theorem}

\anote 各项为非负的级数收敛,当且仅当其部分和构成有界数列。

我们现在将收敛测试转向称为\say{比较测试}。

\begin{theorem}\mbox{}\par
  \begin{enumerate}[label=(\alph*)]
    \item If $|a_n| \le c_n$ for $n \ge N_0$, where $N_0$ is some fixed integer, and if $\Sigma c_n$ converges,
          then $\Sigma a_n$ converges.
    \item If $a_n \ge d_n \ge 0$ for $n \ge N_0$, and if $\Sigma d_n$ diverges, then $\Sigma a_n$ diverges.
  \end{enumerate}
\end{theorem}

注意 (b) 仅作用于级数的非负部分 $a_n$。

\begin{proof}
  给定 $\varepsilon > 0$,根据柯西准则,存在 $N \ge N_0$ 满足 $m \ge n \ge N$ 使得
  \[ \sum_{k=n}^{m} c_k \le \varepsilon \]
  因此
  \[ \Bigg| \sum_{k=n}^{m} a_k \Bigg| \le \sum_{k=n}^{m} |a_k| \le \sum_{k=n}^{m} c_k \le \varepsilon \]
  且 (a) 成立。

  接着,(b) 遵循 (a),如果 $\Sigma a_n$ 收敛,所以 $\Sigma d_n$(注意 (b) 同样遵循 Theorem 3.24)。
\end{proof}

比较测试是非常有用的;为了更效率的使用,我们需要熟悉一些级数的非负项是否为收敛或者发散。

\begin{anote}\mbox{}\par
  (a) 如果 $N_0$ 是某个固定的正整数。$n \ge N_0$ 时 $|a_n| \le c_n$ 而且 $\Sigma a_n$ 也收敛;(b) 如果当 $n \ge N_0$ 时
  $a_n \ge d_n \ge 0$ 而且 $\Sigma d_n$ 发散,那么 $\Sigma a_n$ 也发散。

  对于 (a) 而言,当 $n \ge N_0$ 之后的所有 $n$ 都满足 $|a_n| \le c_n$,那么往后的每个 $n$ 即为
  \begin{gather*}
    |a_{n+1}| \le c_{n+1} \\
    |a_{n+2}| \le c_{n+2} \\
    \cdots \\
    |a_{m}| \le c_{m}
  \end{gather*}
  那么将上述所有不等式相加以后得到的便是
  \[ \sum_{k=n}^{m} |a_k| \le \sum_{k=n}^{m} c_k \]
  而又因为柯西标准的缘故有 $\sum_{k=n}^{m} c_k \le \varepsilon$,同时 $\Bigg| \sum_{k=n}^{m} a_k \Bigg|$ 意为和之绝对值,
  是肯定小于或等于绝对值之和的,因此有了证明中
  \[ \Bigg| \sum_{k=n}^{m} a_k \Bigg| \le \sum_{k=n}^{m} |a_k| \le \sum_{k=n}^{m} c_k \le \varepsilon \]
  这样的不等式。
\end{anote}


\end{document}
