\documentclass[../poma-notes.tex]{subfiles}

\begin{document}

\subsection*{The Root and Ratio Tests}

\begin{theorem}[Root Test]
  Given $\sum_{a_n}$, put $\alpha = \limsup_{n \to \infty} \sqrt[n]{|a_n|}$.
  Then
  \begin{enumerate}[label=(\alph*)]
    \item if $\alpha < 1$, $\Sigma a_n$ converges;
    \item if $\alpha > 1$, $\Sigma a_n$ diverges;
    \item if $\alpha = 1$, the test gives no information.
  \end{enumerate}
\end{theorem}

\begin{proof}
  如果 $\alpha < 1$,我们可以选择 $\beta$ 使得 $\alpha < \beta < 1$,以及一个整数 $N$ 满足
  \[ \sqrt[n]{|a_n|} < \beta \]
  其中 $n \ge N$(根据 Theorem 3.17(b))。即,$n \ge N$ 说明
  \[ |a_n| < \beta^n \]。
  因为 $0 < \beta < 1$,$\Sigma \beta^n$ 收敛,其收敛遵循比较测试。

  如果 $\alpha > 1$,那么在此根据 Theorem 3.17,存在一个序列 $\{n_k\}$ 满足
  \[ \sqrt[n_k]{|a_{n_k}|} \to \alpha \]

  因此 $|a_n| > 1$ 对于无穷的 $n$,条件 $a_n \to 0$ 对于 $\Sigma a_n$ 的收敛而言是必要的,即不成立(Theorem 3.23)。

  证明 (c),考虑以下级数
  \[
    \sum \frac{1}{n}, \ \sum \frac{1}{n^2}
  \]

  对于这些级数 $\alpha = 1$,但是前者发散,后者收敛。
\end{proof}

\anote \textbf{根值审敛法}

\begin{theorem}[Ratio Test]
  The series $\Sigma a_n$
  \begin{enumerate}[label=(\alph*)]
    \item converges if $\limsup_{n \to \infty} |\frac{a_{n+1}}{a_n}| < 1$,
    \item diverges if $|\frac{a_{n+1}}{a_n}| \ge 1$ for all $n \ge n_0$, where $n_0$ is some fixed integer.
  \end{enumerate}
\end{theorem}

\begin{proof}
  如果条件 (a) 成立,我们可以找到 $\beta < 1$,以及一个整数 $N$,满足
  \[ \Biggl|\frac{a_{n+1}}{a_n}\Biggr| < \beta \]
  其中 $n \ge N$。尤其是,
  \begin{align*}
     & |a_{N+1}| < \beta|a_n|,                    \\
     & |a_{N+2}| < \beta|a_{N+1}| < \beta^2|a_n|, \\
     & \dots                                      \\
     & |a_{N+p}| < \beta^p |a_N|.
  \end{align*}
  即
  \[ |a_n| < |a_N|\beta^{-N} \cdot \beta^n \]
  其中 $n \ge N$,且 (a) 遵循比较测试,因为 $\Sigma \beta^n$ 收敛。

  如果 $|a_{n+1}| \ge |a_n|$ 其中 $n \ge n_0$,很容易得出条件 $a_n \to 0$ 并不成立,因此 (b) 成立。

  \textit{注}:$\lim a_{n+1}/a_n = 1$ 说明不了 $\Sigma a_n$ 是收敛的,级数 $\Sigma 1/n$ 与 $\Sigma 1/n^2$ 可以证明。
\end{proof}

\anote \textbf{比值审敛法}

% TODO

\end{document}
