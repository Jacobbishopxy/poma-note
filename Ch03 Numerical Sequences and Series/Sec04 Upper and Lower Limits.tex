\documentclass[../poma-notes.tex]{subfiles}

\begin{document}

\subsection*{Upper and Lower Limits}

\begin{definition}
  Let $\{s_n\}$ be a sequence of real numbers with the following property: For every real $M$ there is an integer $N$ such that $n \ge N$
  implies $s_n \ge M$. We then write
  \[ s_n \to + \infty \]
  Similarly, if for every real $M$ there is an integer $N$ such that $n \ge N$ implies $s_n \le M$, we write
  \[ s_n \to - \infty \]
\end{definition}

值得注意的是我们现在使用的符号 $\to$(在 Definition 3.1 中提出)用于特定类型的发散序列以及收敛序列,但是对于收敛以及极限的定义(Definition 3.1)并没有改变。

\anote $s_n \to \pm \infty$ 的定义。

\begin{definition}
  Let $\{s_n\}$ be a sequence of real numbers. Let $E$ be the set of numbers $x$ (in the extended real number system) such that
  $s_{n_k} \to x$ for some subsequence $\{s_{n_k}\}$. This set $E$ contains all subsequential limits as defined in Definition 3.5, plus
  possibly the numbers $+\infty,\ -\infty$.
\end{definition}

我们现在回忆 Definitions 1.8 与 1.23 且令
\[ s^* = \sup E \]
\[ s_* = \inf E \]
数字 $s^*, s_*$ 分别被称为 $\{s_n\}$ 的 \textit{上极限} 与 \textit{下极限};使用以下标记
\[ \lim_{n \to \infty} \sup s_n = s^*, \qquad \lim_{n \to \infty} \inf s_n = s_* \]

\begin{theorem}
  Let $\{s_n\}$ be a sequence of real numbers. Let $E$ and $s^*$ have the same meaning as in Definition 3.16. Then $s^*$ has the following
  two properties:
  \begin{enumerate}[label=(\alph*)]
    \item $s^* \in E$.
    \item If $x > s^*$, there is an integer $N$ such that $n \ge N$ implies $s_n < x$.
  \end{enumerate}
  Moreover, $s^*$ is the only number with the properties (a) and (b). Of course, an analogous result is true for $s_*$.
\end{theorem}

\begin{proof}
  \begin{enumerate}[label=(\alph*)]
    \item 如果 $s^* = + \infty$,那么 $E$ 则没有上界;因此 $\{s_n\}$ 没有上界,且存在一个子序列 $\{s_{n_k}\}$ 满足 $s_{n_k} \to + \infty$。

          如果 $s^*$ 是实数,那么 $E$ 则有上界,且至少存在一个子序列的极限,因此 (a) 遵循 Theorems 3.7 与 3.8。

          如果 $s^* = - \infty$,那么 $E$ 仅包含一个元素,即 $- \infty$,且没有子序列极限。因此对于任何实数 $M, s_n > M$,至多有一个为 $n$ 的有限数,
          满足 $s_n \to - \infty$。

          上述构成了 (a) 的所有条件。
    \item 假设存在一个数 $x > s^*$,对于无限多个的 $n$ 满足 $s_n \ge x$ 。这种情况下,存在一个数 $y \in E$ 满足 $y \ge x > s^*$,与 $s^*$ 的定义相悖。

          因此 $s^*$ 满足 (a) 与 (b)。

          而对于唯一性,假设存在两个数,$p$ 与 $q$,满足 (a) 与 (b),且假设 $p < q$。选择 $x$ 满足 $p < x < q$。由于 $p$ 满足 (b),则对于 $n \ge N$,
          有 $s_n < x$。但是这样 $q$ 就无法满足 (a) 了。
  \end{enumerate}
\end{proof}

\begin{examples}\mbox{}
  \begin{enumerate}[label=(\alph*)]
    \item Let $\{s_n\}$ be a sequence containing all rationals. Then every real number is a subsequential limit, and
          \[ \lim_{n \to \infty} \sup s_n = + \infty, \qquad \lim_{n \to \infty} \inf s_n = - \infty \].
    \item Let $s_n = (-1^n)/[1+(1/n)] = 1$. Then
          \[ \lim_{n \to \infty} \sup s_n = 1, \qquad \lim_{n \to \infty} \inf s_n = -1 \]
    \item For a real-valued sequence $\{s_n\},\ \lim_{n \to \infty} s_n = s$ if and only if
          \[ \lim_{n \to \infty} \sup s_n = \lim_{n \to \infty} \inf s_n = s \]
  \end{enumerate}
\end{examples}

本节由一个有用的 theorem 作为完结:

\begin{theorem}
  If $s_n \le t_n$ for $n \ge N$, where $N$ is fixed, then
  \[ \lim_{n \to \infty} \inf s_n \le \lim_{n \to \infty} \inf t_n, \]
  \[ \lim_{n \to \infty} \sup s_n \le \lim_{n \to \infty} \sup t_n. \]
\end{theorem}

\end{document}
