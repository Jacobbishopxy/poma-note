\documentclass[../poma-notes.tex]{subfiles}

\begin{document}

\subsection*{Some Special Sequences}

我们现在可以计算一些常见序列的极限了。这些证明都将基于以下观察:如果对于 $n \ge N$ 有 $0 \le x_n \le s_n$,这里 $N$ 为某固定的数,
那么如果 $s_n \to 0$ 则有 $x_n \to 0$。

\begin{theorem}\mbox{}
  \begin{enumerate}[label=(\alph*)]
    \item If $p > 0$, then $\lim_{n \to \infty} \frac{1}{n^p} = 0$.
    \item If $p > 0$, then $\lim_{n \to \infty} \sqrt[n]{p} = 1$.
    \item $\lim_{n \to \infty} \sqrt[n]{n} = 1$.
    \item If $p > 0$ and $\alpha$ is real, then $\lim_{n \to \infty} \frac{n^{\alpha}}{(1 + p)^n} = 0$.
    \item If $|x| < 1$, then $\lim_{n \to \infty} x^n = 0$.
  \end{enumerate}
\end{theorem}

\begin{proof}
  \begin{enumerate}[label=(\alph*)]
    \item 取 $n > (1/\varepsilon)^{1/p}$。(注意这里使用了实数系统的阿基米德性质。)
    \item 如果 $p > 1$,令 $x_n = \sqrt[n]{p} - 1$。那么 $x_n > 0$,且根据二项式定理,
          \[ 1 + nx_n \le (1 + x_n)^n = p \]
          使得
          \[ 0 < x_n \le \frac{p-1}{n} \]
          因此 $x_n \to 0$。如果 $p = 1$,(b) 则无需证明,如果 $0 < p < 1$,则通过倒数可以得出结论。
    \item 令 $x_n = \sqrt[n]{n} - 1$。那么当 $x_n \ge 0$,且根据二项式定理,
          \[ n = (1 + x_n)^n \ge \frac{n(n-1)}{2} x_n^2 \]
          因此
          \[ 0 \le x_n \le \sqrt{\frac{2}{n-1}} \qquad (n \ge 2) \]
    \item 令 $k$ 为一个整数,满足 $k > \alpha, k > 0$。对于 $n > 2k$,
          \[ (1+p)^n>\tbinom{n}{k}p^k=\frac{n(n-1)\cdots(n-k+1)}{k!}p^k>\frac{n^kp^k}{2^kk!}  \]
          因此
          \[ 0 < \frac{n^{\alpha}}{(1+p)^n} < \frac{2^kk!}{p^k} n^{\alpha-k} \qquad (n>2k)\]
          由于 $\alpha - k < 0$,那么根据 (a) 有 $n^{\alpha-k} \to 0$。
    \item 在 (d) 中取 $\alpha = 0$。
  \end{enumerate}
\end{proof}

\begin{anote}\mbox{}\par
  (b) 中的 $0 < p < 1$ 的情况下,利用倒数即令 $x_n = \frac{1}{\sqrt[n]{p}} - 1$,那么
  \begin{gather*}
    (x_n + 1)^n = \frac{1}{p}              \\
    nx_n + 1 \le (x_n + 1)^n = \frac{1}{p} \\
    nx_n + 1 \le \frac{1}{p}               \\
    x_n \le \frac{\frac{1}{p} - 1}{n}      \\
    x_n \le \frac{1-p}{np}
  \end{gather*}
  随着 $n$ 增大 $x_n$ 趋近于 $0$,即 $x_n \to 0$。

  (d) 中的假设 $(1 + p)^n > \tbinom{n}{k} p^k$ 可以分别将 $k=1$ 以及 $k=n$ 带入,对不等式进行验证。当 $k=1$ 时
  \[ (1 + p)^n > \tbinom{n}{1} p^1 = np \]
  显然不等式成立;当 $k=n$ 时
  \[ (1 + p)^n > \tbinom{n}{n} p^n = p^n \]
  不等式也同样成立。那么在 $\tbinom{n}{k} p^k$ 展开后的分子满足
  \begin{gather*}
    n > \frac{n}{2} \\
    (n-1) > \frac{n}{2} \\
    \cdots \\
    (n-k+1) > \frac{n}{2}
  \end{gather*}
  其中最后一项可以简化为 $n > 2k - 2$,而证明中给定的假设是 $n > 2k$,因此同样满足不等式。将上述不等式相乘即
  \[ (n)(n-1)\cdots(n-k+1) > (\frac{n}{2})^k \]
  即
  \[ \frac{(n)(n-1)\cdots(n-k+1)}{k!} p^k > \frac{n^kp^k}{2^kk!} \]
  那么可以得到
  \begin{gather*}
    (1+p)^n > \frac{n^kp^k}{2^kk!} \\
    \frac{2^kk!}{p^k} > \frac{n^k}{(1+p)^n} \\
  \end{gather*}
  将不等式右侧的 $n^k$ 转换为 $n^{k + \alpha - \alpha}$,也就是
  \begin{gather*}
    \frac{2^kk!}{p^k} > \frac{n^{k + \alpha - \alpha}}{(1+p)^n} \\
    \frac{2^kk!}{p^k} > \frac{n^{\alpha}}{(1+p)^n} \cdot n^{k - \alpha} \\
    \frac{n^{\alpha}}{(1+p)^n} < \frac{2^kk!}{p^k} n^{\alpha-k}
  \end{gather*}
  又因为假设了 $k > \alpha$,最后通过 (a) 即 $\lim_{n \to \infty} n^{\alpha - k} = 0$。
\end{anote}

\end{document}
