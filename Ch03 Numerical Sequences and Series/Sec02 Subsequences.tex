\documentclass[../poma-notes.tex]{subfiles}

\begin{document}

\subsection*{Subsequences}

\begin{definition}
  Given a sequence $\{p_n\}$, consider a sequence $\{n_k\}$ of positive integers, such that $n_1 < n_2 < n_3 < \cdots$.
  Then the sequence $\{p_{n_i}\}$ is called a subsequence of $\{p_n\}$. If $\{p_{n_i}\}$ converges, its limit is called
  a \textit{subsequential limit} of $\{p_n\}$.

  It is clear that $\{p_n\}$ converges to $p$ if and only if every subsequence of $\{p_n\}$ converges to $p$.
\end{definition}

\anote 有序列 $\{p_n\}$ 取正整数序列 $\{n_k\}$ 使 $n_1 < n_2 < \dots$,那么序列 $\{p_{n_i}\}$ 便叫做\textbf{子序列},
如果 $\{p_{n_i}\}$ 收敛,那么它的极限叫做 $\{p_n\}$ 的\textbf{部分极限}。序列收敛于 $p$ 当且仅当它的任何子序列收敛于 $p$。

\begin{theorem}
  \begin{enumerate}[label=(\alph*)]
    \item If $\{p_n\}$ is a sequence in a compact metric space $X$, then some subsequence of $\{p_n\}$ converges to a
          point of $X$.
    \item Every bounded sequence in $R^k$ contains a convergent subsequence.
  \end{enumerate}
\end{theorem}

\begin{proof}
  \begin{enumerate}[label=(\alph*)]
    \item 令 $E$ 为 $\{p_n\}$ 的值域。如果 $E$ 是有限的,那么存在一个 $p\in E$ 及一个序列 $\{n_i\}$ 且 $n_1<n_2<n_3<\cdots$
          满足
          \[ p_{n_1} = p_{n_2} = \cdots = p \]
          而子序列 $\{p_{n_i}\}$ 因此收敛于 $p$。

          如果 $E$ 是无限的,根据 Theorem 2.37 所提的 $E$ 拥有一个极限点 $p \in X$。选取 $n_1$ 使得 $d(p,p_{n_1}) < 1$。
          选取 $n_1, \dots, n_{i-1}$,根据 Theorem 2.20 所提的,存在一个整数 $n_i > n_{i-1}$ 满足 $d(p, p_{n_i}) < 1/i$。
          那么 $\{p_{n_i}\}$ 收敛于 $p$。
    \item 遵循 (a),因为 Theorem 2.41 推导每个有界的 $R^k$ 子集存在于 $R^k$ 的一个紧集中。
  \end{enumerate}
\end{proof}

\anote 如果 $\{p_n\}$ 是紧度量空间 $X$ 中的序列,那么 $\{p_n\}$ 有某个子序列收敛到 $X$ 中的某个点;$R^k$ 中的每个有界序列包含
收敛的子序列。

\begin{theorem}
  The subsequential limits of a sequence $\{p_n\}$ in a metric space $X$ form a closed subset of $X$.
\end{theorem}

\begin{proof}
  令 $E^*$ 为 $\{p_n\}$ 所有子序列极限的集合,同时令 $q$ 为 $E^*$ 的一个极限点。我们需要证明 $q \in E^*$。

  选取 $n_1$ 满足 $p_{n_1} \ne q$。(如果没有这样的 $n_1$ 存在,那么 $E^*$ 只有一个点,这样便没有必要证明了。)
  令 $\delta = d(q, p_{n_1})$。假设选取了 $n1, \dots, n_{i-1}$。因为 $q$ 是 $E^*$ 的一个极限点,那么存在一个 $x\in E^*$
  满足 $d(x,q) < 2^{-i} \delta$。因为 $x \in E^*$,存在一个 $n_i > n_{i-1}$ 满足 $d(x,p_{n_i}) < 2^{-i} \delta$。因此
  \[ d(q, p_{n_i}) \le 2^{1-i} \delta \]
  对于 $i = 1,2,3,\dots$。这意味着 $\{p_{n_i}\}$ 收敛于 $q$。因此 $q \in E^*$。
\end{proof}

\anote 度量空间 $X$ 里的序列 $\{p_n\}$ 的部分极限组成 $X$ 的闭子集。

\end{document}
