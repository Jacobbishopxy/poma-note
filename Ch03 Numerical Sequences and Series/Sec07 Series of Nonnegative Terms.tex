\documentclass[../poma-notes.tex]{subfiles}

\begin{document}

\subsection*{Series of Nonnegative Terms}

\begin{theorem}
  If $0 \le x < 1$, then
  \[ \sum_{n=0}^{\infty} x^n = \frac{1}{1-x}. \]
  If $x \ge 1$, the series diverges.
\end{theorem}

\begin{proof}
  如果 $x \ne 1$,
  \[ s_n = \sum_{k=0}^{n} x^k = \frac{1-x^{n+1}}{1-x} \]
  如果令 $n \to \infty$,对于 $x=1$ 而言,有
  \[ 1 + 1 + 1 + \cdots \]
  即明显是发散的。
\end{proof}

在应用中出现的很多案例,级数部分是单调递减的。下面的柯西定理因此特别有趣。该定理显著的特征是 $\{a_n\}$ 的一个相当 \say{薄} 的子序列决定了 $\Sigma a_n$
的收敛或发散。

\begin{theorem}
  Suppose $a_1 \ge a_2 \ge a_3 \ge \cdots \ge 0$. Then the series $\Sigma_{n=1}^{\infty} a_n$ converges if and only if the series
  \begin{equation}
    \sum_{k=0}^{\infty} 2^k a_{2^k} = a_1 + 2a_2 + 4a_4 + 8a_8 + \cdots
  \end{equation}
  converges.
\end{theorem}

\begin{proof}
  根据 Theorem 3.24,足够考虑部分和的边界。令
  \begin{gather*}
    s_n = a_1 + a_2 + \cdots + a_n, \\
    t_k = a_1 + 2a_2 + \cdots + 2^k a_{2^k}.
  \end{gather*}

  对于 $n < 2^k$,有
  \begin{align*}
    \mathcal{} s_n & \le a_1 + (a_2 + a_3) + \cdots + (a_{2^k} + \cdots + a_{2^{k+1}-1}) \\
                   & \le a_1 + 2a_2 + \cdots + 2^k a_{2^k}                               \\
                   & = t_k,
  \end{align*}
  因此
  \begin{equation}
    s_n \le t_k
  \end{equation}
  另一方面,如果 $n > 2^k$,
  \begin{align*}
    \mathcal{} s_n & \ge a_1 + a_2 + (a_3 + a_4) + \cdots + (a_{2^{k-1}+1} + \cdots + a_{2^k}) \\
                   & \ge \frac{1}{2}a_1 + a_2 + 2a_4 + \cdots + 2^{k-1}a_{2^k}                 \\
                   & = \frac{1}{2} t_k,
  \end{align*}
  因此
  \begin{equation}
    2s_n \ge t_k.
  \end{equation}

  根据 (8) 与 (9),序列 $\{s_n\}$ 与 $\{t_k\}$ 要么同时是有界的,要么是同时是无界的。证明完毕。
\end{proof}

\begin{anote}
  令 $a_1 \ge a_2 \ge a_3 \ge \cdots \ge 0$,那么 $\Sigma_{n=1}^{\infty} a_n$ 收敛,当且仅当 $\Sigma_{n=0}^{\infty}2^k a_{2^k}$ 收敛。
\end{anote}

% TODO

\begin{theorem}
  $\sum \frac{1}{n^p}$ converges if $p > 1$ and diverges if $p \le 1$.
\end{theorem}

\begin{proof}
  如果 $p \le 0$,遵循 Theorem 3.23 收敛。如果 $p > 0$,则 Theorem 3.27 是适当的,且将带来级数
  \[ \sum_{k=0}^{\infty} 2^k \cdot \frac{1}{2^{kp}} = \sum_{k=0}^{\infty} 2^{(1-p)k} \]
  现在,$2^{1-p} < 1$ 当且仅当 $1-p < 0$,那么该结果咋遵循了几何级数的比较测试(在 Theorem 3.26 中取 $x = 2^{1-p}$)。
\end{proof}

\anote 若 $p>1$,$\Sigma\frac{1}{n^p}$ 收敛;若 $p\le1$,它就发散。

% TODO

\begin{theorem}
  If $p > 1$,
  \begin{equation}
    \sum_{n=2}^{\infty} \frac{1}{n(\log n)^p}
  \end{equation}
  converges; if $p \le 1$, the series diverges.
\end{theorem}

\begin{remark}
  \say{log n} 意味着 $n$ 在基 $e$ 的对数;$e$ 将在后续 Definition 3.30 中进行定义。我们令级数从 $n = 2$ 开始,因为 $\log 1 = 0$。
\end{remark}

\begin{proof}
  对数函数的单调性(将会在第八章中做详细的讨论)表明 $\{\log n\}$ 递增。因此 $\{1/n \log n\}$ 递减,可以应用 Theorem 3.27 在 (10)上;
  这将给我们一个级数
  \begin{equation}
    \sum_{k=1}^{\infty} 2^k \cdot \frac{1}{2^k(\log 2^k)} =
    \sum_{k=1}^{\infty} \frac{1}{(k \log 2)^p} =
    \frac{1}{(\log 2)^p} \sum_{k=1}^{\infty}\frac{1} {k^p}
  \end{equation}
  且 Theorem 3.29 遵循 Theorem 3.28.

  该过程很明显是连续的。例如,
  \begin{equation}
    \sum_{n=3}^{\infty} \frac{1}{n \log n \log \log n}
  \end{equation}
  是发散的,而
  \begin{equation}
    \sum_{n=3}^{\infty} \frac{1}{n \log n(\log \log n)^2}
  \end{equation}
  是收敛的。
\end{proof}

% TODO

\end{document}
