\documentclass[../poma-notes.tex]{subfiles}

\begin{document}

\subsection*{Addition and Multiplication of Series}

\begin{theorem}
  If $\Sigma a_n = A$, and $\Sigma b_n = B$, then $\Sigma(a_n + b_n) = A + B$, and $\Sigma c a_n = c A$,
  for any fixed c.
\end{theorem}

\begin{proof}
  令
  \[
    A_n = \sum_{k=0}^{n} a_k, \qquad B_n = \sum_{k=0}^{n} b_k.
  \]
  那么有
  \[
    A_n + B_n = \sum_{k=0}^{n}(a_k + b_k).
  \]
  又因为 $\lim_{n\to\infty} A_n = A$ 以及 $\lim_{n\to\infty} B_n = B$,可得
  \[
    \lim_{n\to\infty} (A_n + B_n) = A + B
  \]
  而第二个声明更简单,此处省略。

  因此两个收敛的级数或许可以逐项相加,而得到的级数收敛于两个级数之和。当考虑级数相乘时的情况,则会变得更复杂。这可以有几种方法完成;
  我们应该考虑\say{柯西乘法 Cauchy product}。
\end{proof}

\begin{definition}
  Given $\Sigma a_n$ and $\Sigma b_n$, we put
  \[
    c_n = \sum_{k=0}^{n} a_k b_{n-k} \qquad (n=0,1,2,\dots)
  \]
  and call $\Sigma c_n$ the \textit{product} of the two given series.

  This definition may be motivated as follows. If we take two power series $\Sigma a_n z^n$ and $\Sigma b_n z^n$,
  multiply them term by term, and collect terms containing the same power of $z$, we get
  \begin{align*}
    \begin{split}
      \sum_{n=0}^{\infty} a_n z^n \cdot \sum_{n=0}^{\infty} b_n z^n & = (a_0+a_1z+a_2z^2+\cdots)(b_0+b_1z+b_2z^2+\cdots) \\
      & = a_0b_0 + (a_0b_1 + a_1b_0)z + (a_0b_2+a_1b_1+a_2b_0)z^2 + \cdots \\
      & = c_0 + c_1z + c_2z^2 + \cdots
    \end{split}
  \end{align*}
  Setting $z=1$, we arrive at the above definition.
\end{definition}

\begin{example}
  If
  \[
    A_n = \sum_{k=0}^{n} a_k, \qquad B_n = \sum_{k=0}^{n} b_k, \qquad C_n = \sum_{k=0}^{n} c_k,
  \]
  and $A_n \to A,\ B_n \to B$, then it is not at all clear that $\{C_n\}$ will converge to $AB$, since we do not
  have $C_n = A_n B_n$. The dependence of $\{C_n\}$ on $\{A_n\}$ and $\{B_n\}$ is quite a complicated one (see
  the proof of Theorem 3.50). We shall now show that the product of two convergent series may actually diverge.

  The series
  \[
    \sum_{n=0}^{\infty}\frac{(-1)^n}{\sqrt{n+1}} = 1-\frac{1}{\sqrt{2}}+\frac{1}{\sqrt{3}}-\frac{1}{\sqrt{4}}+\cdots
  \]
  converges (Theorem 3.43). We form the product of this series with itself and obtain
  \begin{multline*}
    \qquad \sum_{n=0}^{\infty} c_n = 1-(\frac{1}{\sqrt{2}}+\frac{1}{\sqrt{2}})+
    (\frac{1}{\sqrt{3}}+\frac{1}{\sqrt{2}\sqrt{2}}+\frac{1}{\sqrt{3}}) \\
    -(\frac{1}{\sqrt{4}}+\frac{1}{\sqrt{3}\sqrt{2}}+\frac{1}{\sqrt{2}\sqrt{3}}+\frac{1}{\sqrt{4}})+\cdots
    \qquad
  \end{multline*}
  so that
  \[
    c_n = (-1)^n \sum_{k=0}^{n} \frac{1}{\sqrt{(n-k+1)(k+1)}}
  \]
  Since
  \[
    (n-k+1)(k+1) = \biggl(\frac{n}{2}+1\biggr)^2 - \biggl(\frac{n}{2}-k\biggr)^2 \le \biggl(\frac{n}{2}+1\biggr)^2
  \]
  we have
  \[
    |c_n| \ge \sum_{k=0}^{n} \frac{2}{n+2} = \frac{2(n+1)}{n+2}
  \]
  so that the condition $c_n \to 0$, which is necessary for the convergence of $\Sigma c_n$, is not satisfied.
\end{example}
\newpage

\begin{anote}
  级数 $\Sigma \frac{(-1)^n}{\sqrt{n+1}}$ 与自身的乘积,以下 $a_n$ 与 $b_n$ 皆为 $\Sigma \frac{(-1)^n}{\sqrt{n+1}}$:
  \begin{align*}
    \begin{split}
      \sum_{n=0}^{\infty} c_n & = \sum_{n=0}^{\infty} a_n \cdot \sum_{n=0}^{\infty} b_n \\
      & = a_0b_0 + (a_0b_1 + a_1b_0) + (a_0b_2+a_1b_1+a_2b_0) + \cdots \qquad \text{(a)} \\
      & = \sum_{n=0}^{\infty} \frac{(-1)^n}{\sqrt{n+1}} \cdot \sum_{n=0}^{\infty} \frac{(-1)^n}{\sqrt{n+1}} \\
      & = \frac{(-1)^0}{\sqrt{0+1}} \cdot \frac{(-1)^0}{\sqrt{0+1}} + (\frac{(-1)^0}{\sqrt{0+1}}
      \cdot \frac{(-1)^1}{\sqrt{1+1}} + \frac{(-1)^1}{\sqrt{1+1}} \cdot \frac{(-1)^0}{\sqrt{0+1}}) + \\
      & \qquad (\frac{(-1)^0}{\sqrt{0+1}} \cdot \frac{(-1)^2}{\sqrt{2+1}} + \frac{(-1)^1}{\sqrt{1+1}} \cdot
      \frac{(-1)^1}{\sqrt{1+1}} + \frac{(-1)^2}{\sqrt{2+1}} \cdot \frac{(-1)^0}{\sqrt{0+1}}) + \cdots \\
      & = 1 - (\frac{1}{\sqrt{2}}+\frac{1}{\sqrt{2}}) + (\frac{1}{\sqrt{3}} + \frac{1}{\sqrt{2}\sqrt{2}} +
      \frac{1}{\sqrt{3}}) - (\frac{1}{\sqrt{4}} + \frac{1}{\sqrt{3}\sqrt{2}} + \frac{1}{\sqrt{2}\sqrt{3}} +
      \frac{1}{\sqrt{4}}) \\ & \qquad + \cdots
    \end{split}
  \end{align*}

  那么根据 Definition 3.48,
  \begin{align*}
    \begin{split}
      c_n & = \sum_{k=0}^{n} a_k b_{n-k} \\
      & = \sum_{k=0}^{n} \frac{(-1)^k}{\sqrt{k+1}} \cdot \frac{(-1)^{n-k}}{\sqrt{n-k+1}} \\
      & = \sum_{k=0}^{n} \frac{(-1)^n}{\sqrt{(k+1)(n-k+1)}} \\
      & \ge (-1)^n \sum_{k=0}^{n} \frac{2}{n+2} \qquad \text{(b)}
    \end{split}
  \end{align*}

  那么取绝对值后有
  \begin{align*}
    \begin{split}
      |c_n| & \ge \sum_{k=0}^{n} \frac{2}{n+2}
    \end{split}
  \end{align*}

  显然 $c_n \to 0$ 不成立,那么 $\Sigma c_n$ 则无法收敛(Theorem 3.23)。

  注:
  (a):Definition 3.48 中,$z = 1$ 的情况;
  (b):根据 $(n-k+1)(k+1) = (\frac{n}{2}+1)^2 - (\frac{n}{2}-k)^2 \le (\frac{n}{2}+1)^2$。
\end{anote}

\newpage
\begin{theorem}
  Suppose
  \begin{enumerate}[label=(\alph*)]
    \item $\sum_{n=0}^{\infty} a_n$ converges absolutely,
    \item $\sum_{n=0}^{\infty} a_n = A$,
    \item $\sum_{n=0}^{\infty} b_n = B$,
    \item $c_n = \sum_{n=0}^{\infty} a_k b_{n-k} \qquad (n=0,1,2,\dots)$.
  \end{enumerate}
  Then
  \[
    \sum_{n=0}^{\infty} c_n = AB.
  \]

  That is, the product of two convergent series converges, and to the right value, if at least one of the two series converges absolutely.
\end{theorem}

\begin{proof}
  令
  \[
    A_n = \sum_{k=0}^{n} a_k,\qquad B_n = \sum_{k=0}^{n} b_k,\qquad C_n = \sum_{k=0}^{n} c_k,\qquad \beta_n = B_n - B
  \]
  那么有
  \begin{equation*}
    \begin{split}
      C_n & = a_0 b_0 + (a_0 b_1 + a_1 b_0) + \cdots + (a_0 b_n + a_1 b_{n-1} + \cdots + a_n b_0) \\
      & = a_0 B_n + a_1 B_{n-1} + \cdots + a_n B_0 \\
      & = a_0 (B + \beta_n) + a_1 (B + \beta_{n-1}) + \cdots + a_n {B + \beta_0} \\
      & = A_n B + a_0 \beta_n + a_1 \beta_{n-1} + \cdots + a_n \beta_0
    \end{split}
  \end{equation*}
  令
  \[
    \gamma_n = a_0 \beta_n + a_1 \beta_{n-1} + \cdots + a_n \beta_0
  \]

  我们希望得到 $C_n \to AB$。由于 $A_n B \to AB$,它足够得到
  \begin{equation}
    \lim_{n\to\infty} \gamma_n = 0
  \end{equation}
  令
  \[
    \alpha = \sum_{n=0}^{\infty} |a_n|
  \]
  【这里用到了 (a)。】给定 $\varepsilon > 0$。根据(c),$\beta_n \to 0$。因此我们可以选择 $N$ 使得对于 $n \ge N$ 满足
  $|\beta_n| \le \varepsilon$,其中
  \begin{align*}
    \begin{split}
      |\gamma_n| & \le |\beta_0 a_n + \cdots + \beta_N a_{n-N}| + |\beta_{N+1} a_{n-N-1} + \cdots + \beta_n a_0| \\
      & \le |\beta_0 a_n + \cdots + \beta_N a_{n-N}| + \varepsilon\alpha
    \end{split}
  \end{align*}
  保持 $N$ 固定,令 $n \to \infty$,得到
  \[
    \limsup_{n\to\infty} |\gamma_n| \le \varepsilon\alpha
  \]
  因为 $k \to \infty$ 时有 $a_k \to 0$。而 $\varepsilon$ 是随机的,式(21)成立。
\end{proof}

\begin{anote}
  \textbf{重点}:$AB$ 至少一个绝对收敛。

  关于 $C_n$ 的计算:
  \begin{align*}
    \begin{split}
      C_n & = \aul[red]{a_0} b_0 + (\aul[red]{a_0} b_1 + \aul[blue]{a_1} b_0) + \cdots +
      (\aul[red]{a_0} b_n + \aul[blue]{a_1} b_{n-1} + \cdots + a_n b_0) \\
      & = \aul[red]{a_0}(b_0 + b_1 + \cdots + b_n) + \aul[blue]{a_1}(b_0 + \cdots + b_{n-1}) +
      \cdots + a_n (b_0) \\
      & = a_0 B_n + a_1 B_{n-1} + \cdots + a_n B_0 \\
      & = \aul[teal]{a_0} (B + \beta_n) + \aul[teal]{a_1} (B + \beta_{n-1}) + \cdots +
      \aul[teal]{a_n} {B + \beta_0} \\
      & = \aul[teal]{A_n} B + a_0 \beta_n + a_1 \beta_{n-1} + \cdots + a_n \beta_0
    \end{split}
  \end{align*}

  关于 $\gamma_n$ 的计算:
  \begin{align*}
    \begin{split}
      \gamma_n & = a_0 \beta_n + a_1 \beta_{n-1} + \cdots + a_n \beta_0 \\
      & = \beta_0 a_n + \beta_1 a_{n-1} + \cdots + \beta_N a_{n-N} + \beta_{N-1} a_{n-(N+1)} +
      \cdots + \beta_n a_0 \\
      |\gamma_n| & = |\beta_0 a_n + \beta_1 a_{n-1} + \cdots + \beta_N a_{n-N}| +
      |\beta_{N-1} a_{n-(N+1)} + \cdots + \beta_n a_0| \\
      & \le |\beta_0 a_n + \beta_1 a_{n-1} + \cdots + \beta_N a_{n-N}| + \varepsilon\alpha
    \end{split}
  \end{align*}
\end{anote}

\begin{theorem}
  If the series $\Sigma a_n, \Sigma b_n, \Sigma c_n$ converge to $A, B, C$, and $c_n = a_0 b_n + \cdots + a_n b_0$, then $C = AB$.
\end{theorem}

这里没有关于绝对收敛的假设。我们将在 Theorem 8.2 完成之后给到一个简单的证明(其依赖于幂级数的连续性)。

\end{document}
