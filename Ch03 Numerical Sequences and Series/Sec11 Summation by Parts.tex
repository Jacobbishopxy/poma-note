\documentclass[../poma-notes.tex]{subfiles}

\begin{document}

\subsection*{Summation by Parts}

\begin{theorem}
  Given two sequences $\{a_n\},\ \{b_n\}$, put
  \[ A_n = \sum_{k=0}^{n} a_k \]
  if $n \ge 0$; put $A_{-1} = 0$. Then, if $0 \le p \le q$, we have
  \begin{equation}
    \sum_{n=p}^{q} a_n b_n = \sum_{n=p}^{q-1} A_n(b_n-b_{n+1}) + A_q b_q - A_{p-1}b_p
  \end{equation}
\end{theorem}

\begin{proof}
  \[
    \sum_{n=p}^{q} a_n b_n = \sum_{n=p}^{q}(A_n - A_{n-1})b_n = \sum_{n=p}^{q} A_n b_n - \sum_{n=p-1}^{q-1}A_n b_{n+1}
  \]
  式(20),被称为\say{部分和式 partial summation formula},在探索带有 $\Sigma a_n b_n$ 项的级数时非常有用,特别是当 $\{b_n\}$
  是单调的情况下。
\end{proof}

\begin{anote}
  \begin{align*}
    \begin{split}
      \sum_{n=p}^{q} a_n b_n & = \sum_{n=p}^{q} (A_n - A_{n-1}) b_n \\
      & = \sum_{n=p}^{q} A_n b_n - \sum_{n=p}^{q} A_{n-1} b_n \\
      & = \sum_{n=p}^{q-1} A_n b_n + A_q b_q - \sum_{n=p}^{q} A_{n-1} b_n \\
      & = \sum_{n=p}^{q-1} A_n b_n + A_q b_q - \sum_{n=p-1}^{q-1} A_n b_{n+1} \\
      & = \sum_{n=p}^{q-1} A_n b_n + A_q b_q - \sum_{n=p}^{q-1} A_n b_{n+1} - A_{p-1} b_p \\
      & = \sum_{n=p}^{q-1} A_n (b_n + b_{n+1}) + A_q b_q - A_{p-1} b_p \\
    \end{split}
  \end{align*}

  其中第 3 步的第三项 $\sum_{n=p}^{q} A_{n-1} b_n$ 转换至第四步的过程中,可将 $n$ 视为 $m+1$,那么就有
  \begin{align*}
    \begin{split}
      \sum_{n=p}^{q} A_{n-1} b_n & = \sum_{m+1=p}^{q} A_{m+1-1} b_{m+1} \\
      & = \sum_{m=p-1}^{q-1} A_m b_{m+1}
    \end{split}
  \end{align*}
\end{anote}

\begin{theorem}
  Suppose
  \begin{enumerate}[label=(\alph*)]
    \item the partial sums $A_n$ of $\Sigma a_n$ form a bounded sequence;
    \item $b_0 \ge b_1 \ge b_2 \ge \cdots$ ;
    \item $\lim_{n\to\infty} b_n = 0$.
  \end{enumerate}
  Then $\Sigma a_n b_n$ converges.
\end{theorem}

\begin{proof}
  选取 $M$ 对所有 $n$ 满足 $|A_n| \le M$。给定 $\varepsilon > 0$,存在一个整数 $N$ 使得 $b_N \le (\varepsilon/2M)$。
  对于 $N \le p \le q$,有
  \begin{align*}
    \begin{split}
      \Biggl|\sum_{n=p}^{q} a_n b_n\Biggr| & = \Biggl|\sum_{n=p}^{q-1} A_n(b_n - b_{n+1}) + A_q b_q - A_{p-q} b_p\Biggr| \\
      & \le M \Biggl|\sum_{n=p}^{q-1}(b_n-b_{n+1}) + b_q + b_p\Biggr| \\
      & = 2M b_p \le 2M b_N \le \varepsilon
    \end{split}
  \end{align*}
  收敛现在遵循柯西准则。注意第一个不等式中依赖了 $b_n - b_{n+1} \ge 0$。
\end{proof}

\begin{theorem}
  Suppose
  \begin{enumerate}[label=(\alph*)]
    \item $|c_1| \ge |c_2| \ge |c_3| \ge \cdots$;
    \item $c_{2m-1} \ge 0,\ c_{2m} \le 0 \qquad (m=1,2,3,\cdots)$;
    \item $\lim_{n\to\infty} c_n = 0$.
  \end{enumerate}
  Then $\Sigma c_n$ converges.

  满足 (b) 的级数也被称为\say{交错级数 alternating series};该定理也熟知为莱布尼兹定理。
\end{theorem}

\begin{proof}
  应用 Theorem 3.42,以及 $a_n = (-1)^{n+1},\ b_n = |c_n|$。
\end{proof}

\begin{theorem} Suppose the radius of convergence of $\Sigma c_n z^n$ is $1$, and suppose $c_0\ge c_1\ge c_2\ge\cdots,
    \lim_{n\to\infty}c_n = 0$. Then $\Sigma c_n z^n$ converges at every point on the circle $|z|=1$,
  except possibly at $z=1$.
\end{theorem}

\begin{proof}
  令 $a_n = z^n,\ b_n = c_n$。Theorem 3.42 的假设满足了,因为
  \[
    |A_n| = \Biggl|\sum_{m=0}^{n}z^m\Biggr| = \Biggl|\frac{1-z^{n+1}}{1-z}\Biggr| \le \frac{2}{|1-z|}
  \]
  如果 $|z| = 1,\ z \ne 1$
\end{proof}

\end{document}
