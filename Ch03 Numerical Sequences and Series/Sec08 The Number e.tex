\documentclass[../poma-notes.tex]{subfiles}

\begin{document}

\subsection*{The Number e}

\begin{definition}
  \[ e = \sum_{n=0}^{\infty} \frac{1}{n!} \]
  这里 $n! = 1 \cdot 2 \cdot 3 \cdots n$ 如果 $n \ge 1$,且 $0! = 1$。

  因为
  \begin{align*}
    \mathcal{} s_n & = 1 + 1 + \frac{1}{1 \cdot 2} + \frac{1}{1 \cdot 2 \cdot 3} + \cdots + \frac{1}{1 \cdot 2 \cdot n} \\
                   & < 1 + 1 + \frac{1}{2} + \frac{1}{2^2} +  \cdots + \frac{1}{2^{n-1}}                                \\
                   & < 3
  \end{align*}
  级数收敛,因此定理成立。实际上,该级数收敛的非常迅速,使得我们可以计算 $e$ 时获得很大的精确度。
\end{definition}

同样 $e$ 也可以被其他的极限过程所定义:

\begin{theorem}
  \[ \lim_{n \to \infty} \Biggl( 1 + \frac{1}{n} \Biggr)^n = e. \]
\end{theorem}

\begin{proof}
  令
  \[ s_n = \sum_{k=0}^{n} \frac{1}{k!}, \qquad t_n = \Biggl( 1 + \frac{1}{n} \Biggr)^n \]

  根据二项式定理,
  \begin{multline*}
    t_n = 1+1+\frac{1}{2!}\Biggl(1-\frac{1}{n}\Biggr)+\frac{1}{3!}\Biggl(1-\frac{1}{n}\Biggr)\Biggl(1-\frac{2}{n}\Biggr)+\cdots \\
    + \frac{1}{n!}\Biggl(1-\frac{1}{n}\Biggr)\Biggl(1-\frac{2}{n}\Biggr)\cdots\Biggl(1-\frac{n-1}{n}\Biggr)
  \end{multline*}

  因此 $t_n \le s_n$,所以根据 Theorem 3.19 有
  \begin{equation}
    \limsup_{n \to \infty} t_n \le e,
  \end{equation}

  接下来如果 $n \ge m$,
  \[
    t_n \ge 1 + 1 + \frac{1}{2!}\Biggl(1-\frac{1}{n}\Biggr) + \cdots
    + \frac{1}{m!}\Biggl(1-\frac{1}{n}\Biggr) \cdots \Biggl(1-\frac{m-1}{n}\Biggr)
  \]

  令 $n \to \infty$,保持 $m$ 固定。我们有
  \[ \liminf_{n \to \infty} t_n \ge 1 + 1 + \frac{1}{2!} + \cdots + \frac{1}{m!} \]
  使得
  \[ s_m \le \liminf_{n \to \infty} t_n \]

  令 $m \to \infty$,我们最终可以获得
  \begin{equation}
    e \le \liminf_{n \to \infty} t_n
  \end{equation}

  定理遵循 (14) 以及 (15)。

  那么级数 $\sum \frac{1}{n!}$ 的急速收敛可以被如下估计:如果 $s_n$ 具有上述的意义,则有
  \begin{align*}
    \mathcal{} e - s_n & = \frac{1}{(n+1)!} + \frac{1}{(n+2)!} + \frac{1}{(n+3)!} + \cdots                          \\
                       & < \frac{1}{(n+1)!}\Biggl\{1+\frac{1}{n+1}+\frac{1}{(n+1)^2}+\cdots\Biggr\} = \frac{1}{n!n}
  \end{align*}
  使得
  \begin{equation}
    0 < e - s_n < \frac{1}{n!n}
  \end{equation}

  因此比如说 $s_{10}$,近似 $e$ 的错误小于 $10^{-7}$。不等式 (16) 也具有理论上的意义,因为它让我们更容易证明 $e$ 的无理性。
\end{proof}

% TODO

\begin{theorem}
  $e$ is irrational.
\end{theorem}

\begin{proof}
  假设 $e$ 是有理的。那么有 $e = p/q$,且 $p$ 与 $q$ 是正整数。根据 (16),
  \begin{equation}
    0 < q!(e - s_q) < \frac{1}{q}
  \end{equation}
  根据假设,$q!e$ 是一个整数。由于
  \[ q!s_q = q!(1 + 1 + \frac{1}{2!} + \cdots + \frac{1}{q!}) \]
  是一个整数,可知 $q!(e-s_q)$ 是一个整数。

  由于 $q \ge 1$,(17) 表明存在一个位于 0 与 1 之间的整数,有此得到了悖论。
\end{proof}

% TODO

\end{document}
