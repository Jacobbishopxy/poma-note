\documentclass[../poma-notes.tex]{subfiles}

\begin{document}

\subsection*{Connected Sets}

\begin{definition}
  Two subsets $A$ and $B$ of a metric space $X$ are said to be \textit{separated} if both $A \cap \overline{B}$
  and $\overline{A} \cap B$ are empty, i.e., if no point of $A$ lies in the closure of $B$ and no point of $B$
  lies in the closure of $A$.

  A set $E \subset X$ is said to be \textit{connected} if $E$ is \textit{not} a union of two nonempty separated
  sets.
\end{definition}

\begin{anote}
  度量空间 $X$ 的两个子集 $A$ 与 $B$ 被称为\textbf{分离的 separated} 如果 $A \cap \overline{B}$ 与
  $\overline{A} \cap B$ 皆为空集;与分离相反的概念为\textbf{连通的 connected}。
\end{anote}

\begin{remark}
  Separated sets are of course disjoint, but disjoint sets need not be separated. For example, the interval $[0,1]$
  and the segment $(1,2)$ are \textit{not} separated, since $1$ is a limit point of $(1,2)$. However, the segments
  $(0,1)$ and $(1,2)$ \textit{are} separated.
\end{remark}

\begin{anote}
  分离的两个集是不相交的,但不相交的集合不一定是分离的。例如 $[0,1]$ 与 $(1,2)$ 不是分离的。
\end{anote}


线的连通子集拥有一个特殊的简单结构:

\begin{theorem}
  A subset $E$ of the real line $R^1$ is connected if and only if it has the following property: If $x\in E, y\in E$,
  and $x<z<y$, then $z\in E$.
\end{theorem}

\begin{proof}
  如果存在 $x \in E,\ y \in E$,以及某些 $z \in (x,y)$ 使得 $z \notin E$,那么有 $E = A_z \cup B_z$ 其中
  \[A_z = E \cap (-\infty,z), \quad B_z = E \cap (z, \infty)\]
  因为 $x \in A_z$ 以及 $y \in B_z$,$A$ 与 $B$ 为非空。因为 $A_z \subset (-\infty,z)$ 以及 $B_z \subset (z,\infty)$,
  它们是分开着的。因此 $E$ 不为连接。

  接下来是反着证明,假设 $E$ 不为连接。那么不存在非空并分开的集合 $A$ 与 $B$ 使得 $A \cup B = E$。令 $x \in A,\ y \in B$,且
  假设(而不失去普遍性)$x < y$。定义
  \[z = \sup(A \cap [x, y])\]
  根据 Theorem 2.28,$z \in \overline{A}$;因此 $z \notin B$。特别是 $x \le z < y$。

  如果 $z \notin A$,它遵循 $x < z < y$ 且 $z \notin E$。

  如果 $z \in A$,那么 $z \notin \overline{B}$,因此存在 $z_1$ 使得 $z < z_1 < y$ 且 $z_1 \notin B$。那么 $x < z_1 < y$
  且 $z_1 \notin E$。
\end{proof}

\begin{anote}
  实数集 $R^1$ 的子集 $E$ 是连通的,当且仅当:如果 $x \in E,\ y \in E$ 且 $x < z < y$,那么 $z \in E$。
\end{anote}


\end{document}
