\documentclass[../poma-notes.tex]{subfiles}

\begin{document}

\subsection*{Connected Sets}

\begin{definition}
  Two subsets $A$ and $B$ of a metric space $X$ are said to be \textit{separated} if both $A \cap \overline{B}$
  and $\overline{A} \cap B$ are empty, i.e., if no point of $A$ lies in the closure of $B$ and no point of $B$
  lies in the closure of $A$.

  A set $E \subset X$ is said to be \textit{connected} if $E$ is \textit{not} a union of two nonempty separated
  sets.
\end{definition}

\anote

\begin{remark}
  Separated sets are of course disjoint, but disjoint sets need not be separated. For example, the interval $[0,1]$
  and the segment $(1,2)$ are \textit{not} separated, since $1$ is a limit point of $(1,2)$. However, the segments
  $(0,1)$ and $(1,2)$ \textit{are} separated.
\end{remark}

\anote

线的连通子集拥有一个特殊的简单结构:

\begin{theorem}
  A subset $E$ of the real line $R^1$ is connected if and only if it has the following property: If $x\in E, y\in E$,
  and $x<z<y$, then $z\in E$.
\end{theorem}

\end{document}
