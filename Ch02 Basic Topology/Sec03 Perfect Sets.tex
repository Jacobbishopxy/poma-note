\documentclass[../poma-notes.tex]{subfiles}

\graphicspath{\subfix{../images/}}

\begin{document}

\subsection*{Perfect Sets}

\begin{theorem}
  Let $P$ be a nonempty perfect set in $R^k$. Then $P$ is uncountable.
\end{theorem}

\begin{proof}
  因为 $P$ 有极限点,$P$ 必须是无限的。假设 $P$ 是可数的,那么通过 $\mathbf{x}_1,\mathbf{x}_2,\mathbf{x}_3,\dots$
  来表示 $P$。随之可以构建一个数列 $\{V_n\}$ 来表示邻域。

  令 $V_1$ 为 $\mathbf{x}_1$ 的任意邻域,使得 $V_n \cap P$ 不为空。如果 $V_1$ 包含了所有的 $\mathbf{y} \in R^k$ 使得
  $|\mathbf{y} - \mathbf{x}_1| < r$,$V_1$ 的闭包 $\overline{V}_1$ 则是所有 $\mathbf{y} \in R^k$ 的集合使得
  $|\mathbf{y} - \mathbf{x}_1| \le r$。

  假设 $V_n$ 被构建了,那么 $V_n \cap P$ 不为空。因为 $P$ 的所有点是 $P$ 的极限点,存在一个邻域 $V_{n+1}$ 使得
  \begin{enumerate*}[label=(\roman*)]
    \item $\overline{V}_{n+1} \subset V_n$,
    \item $\mathbf{x} \notin \overline{V}_{n+1}$,
    \item $V_{n+1} \cap P$ 非空。
  \end{enumerate*}
  根据 (iii),$V_{n+1}$ 满足我们的归纳假设,同时构建可以进行。

  令 $K_n = \overline{V}_n \cap P$。因为 $\overline{V}_n$ 是闭且有界的,$\overline{V}$ 是紧的。因为
  $\mathbf{x}_n \notin K_{n+1}$,$P$ 里没有点在 $\cap_1^{\infty} K_n$ 中。因为 $K_n \subset P$,这意为
  $\cap_1^{\infty} K_n$ 是空的。但是根据 (iii),每个 $K_n$ 是非空的,且根据 (i),$K_n \subset K_{n+1}$;这与 Theorem 2.36
  的 Corollary 相悖。
\end{proof}

\begin{anote}
  \begin{itemize}
    \item [\textbf{定理}] $P$ 为 $R^k$ 内的非空完全集,那么 $P$ 是不可数的。
    \item [\textbf{回顾}] 完全性 perfect 在 Definition 2.18 (h) 中提到,\say{$E$ is \textit{perfect} if $E$ is closed and
            if every point of $E$ is a limit point of $E$}。根据 Wikipedia 对完全集的表述:
          \begin{quote}
            In general topology, a subset of a topological space is \textbf{perfect} if it is closed and has no isolated
            points. Equivalently: the set $S$ is perfect if $S = S'$, where $S'$ denotes the set of all limit points of
            $S$, also known as the derived set of $S$.
          \end{quote}
          相较于闭集,完全集多了一条定义,即集合上的点都是集合的极限点。
    \item [\textbf{证明}] 完全集也是闭集,因此有极限点;又根据 Theorem 2.20 衍生的 Corollary,\say{A finite point set has no
            limit points},即完全集不是有限集合。
          构建的 $\{V_n\}$ 邻域数列与假设中的可数集 $P$ 的 $\mathbf{x}_1,\mathbf{x}_2,\mathbf{x}_3,\dots$ 一一对应。
          那么对于 $V_1$ 这个邻域而言,其中包含的 $\mathbf{y} \in R^k$ 点满足 $|\mathbf{y} - \mathbf{x}_1| < r$(用二维空间对
          $V_1$ 邻域与 $\mathbf{x}_1$ 的关系理解就很简单了:某一点 $x$ 为圆心 $r$ 为半径构成的圆,圆中所有的点构成了 $V$;那么闭包
          $\overline{V}_1$ 则是包含了圆周长上的所有点)。
          如果在一维空间来表示 $\mathbf{x}_n, \mathbf{x}_{n+1}$ 与 $V_n, V_{n+1}$ 之间的关系,那么如图所示:
          \begin{center}
            \begin{tikzpicture}
              \begin{axis}[
                  xmin=-10, xmax=10,
                  axis x line*=middle,
                  hide y axis,
                  ymin=-5,ymax=5,
                  xtick=\empty,
                  xticklabels=\empty,
                ]

                \draw[color=black] (5.1,2.5) node {$\overline{V}_{n+1} \subset V_n $};
                \draw [line width=.5pt,dash pattern=on 1pt off 2pt](-4,0) ellipse (11 and 3.3);
                \draw [line width=.5pt,dash pattern=on 1pt off 2pt](4,0) ellipse (2 and 1.5);

                \begin{scriptsize}
                  \draw [fill=black] (-4,0) circle (1pt);
                  \draw[color=black] (-4,-0.5) node {$\mathbf{x}_n$};
                  \draw[color=black] (6.5,-1.5) node {$V_n$};

                  \draw [fill=black] (4,0) circle (1pt);
                  \draw[color=black] (4,-0.5) node {$\mathbf{x}_{n+1}$};
                  \draw[color=black] (4,-1.5) node {$V_{n+1}$};
                \end{scriptsize}
              \end{axis}
            \end{tikzpicture}
          \end{center}
          即表示出了 $\overline{V}_{n+1} \subset V_n$,且 $\mathbf{x}_n \notin \overline{V}_{n+1}$。
          当令 $K_n = \overline{V}_n \cap P$,那么有 $K_{n+1} = \overline{V}_{n+1} \cap P$。同时,根据 Theorem 2.41,当一个集合既
          闭又有界时,该集合为紧集。因此 $\overline{V}_n$ 为紧集。由上面的 $\mathbf{x}_n \notin \overline{V}_{n+1}$ 与
          $K_{n+1} = \overline{V}_{n+1} \cap P$ 可知 $\mathbf{x}_n \notin K_{n+1}$。那么在一维的图中可知 $K_n\cap K_{n+1}=K_{n+1}$,
          那么当 $n$ 趋近无穷时,可知 $\mathbf{x} \notin \cap_1^{\infty} K_n$,即证明中的 $P$ 中没有点在 $\cap_1^{\infty}$;而又因
          $K_n \subset P$,那么 $\cap_1^{\infty} K_n$ 只能为空。然而根据 (iii) 与 (i) 所知 $K_n$ 不是空集,那么根据 Theorem 2.36 的
          Corollary,\say{如果 $\{K_n\}$ 是非空紧集的数列且 $K_n \supset K_{n+1}\ (n=1,2,3,\dots)$,那么 $\cap_1^{\infty} K_n$
            不为空},与推导相悖,因此 $P$ 是不可数的,证明完毕。
  \end{itemize}
\end{anote}

\begin{corollary}
  Every interval $[a,b]\ (a<b)$ is uncountable. In particular, the set of all real numbers is uncountable.
\end{corollary}

\anote 任何区间都是不可数的,尤其是所有实数的集合都是不可数的。

\begin{namedtheorem}[The Cantor set]
  \normalfont
  The set which we are now going to construct shows that there exist perfect sets in $R^1$ which contain no segment.

  Let $E_0$ be the interval $[0,1]$. Remove the segment $(\frac{1}{3},\frac{2}{3})$, and let $E_1$ be the union of
  the intervals
  \[ [0,\frac{1}{3}]\ [\frac{2}{3},1]. \]
  Remove the middle thirds of these intervals, and let $E_2$ be the union of the intervals
  \[ [0,\frac{1}{9}],\ [\frac{2}{9},\frac{3}{9}],\ [\frac{6}{9},\frac{7}{9}],\ [\frac{8}{9},1]. \]
  Continuing in this way, we obtain a sequence of compact sets $E_n$, such that
  \begin{enumerate}[label=(\alph*)]
    \item $E_1 \supset E_2 \supset E_3 \supset \cdots$;
    \item $E_n$ is the union of $2^n$ intervals, each of length $3^{-n}$.
  \end{enumerate}

  The set
  \[ P = \bigcap\limits_{n=1}^{\infty} E_n \]
  is called the \textit{Cantor set}. $P$ is clearly compact, and Theorem 2.36 shows that $P$ is not empty.

  No segment of the form
  \begin{equation}
    \Biggl( \frac{3k+1}{3^m}, \frac{3k+2}{3^m} \Biggr),
  \end{equation}
  where $k$ and $m$ are positive integers, has a pint in common with $P$. Since every segment $(\alpha, \beta)$
  contains a segment of the form (24), if
  \[ 3^{-m} < \frac{\beta - \alpha}{6}, \]
  $P$ contains no segment.

  To show that $P$ is perfect, it is enough to show that $P$ contains no isolated point. Let $x \in P$, and let $S$
  be any segment containing $x$. Let $I_n$ be that interval of $E_n$ which contains $x$. Choose $n$ large enough,
  so that $I_n \subset S$. Let $x_n$ be an endpoint of $I_n$, such that $x_n \ne x$.

  It follows from the construction of $P$ that $x_n \in P$. Hence $x$ is a limit point of $P$, and $P$ is perfect.

  One of the most interesting properties of the Cantor set is that it provides us with an example of an uncountable
  set of measure zero (the concept of measure will be discussed in Chap. 11).
\end{namedtheorem}

\begin{anote}
  Cantor 集说明了 $R^1$ 中存在没有区间的完全集。Cantor 集在构造的过程中,每次去掉的都是开区间。
  \href{https://zhuanlan.zhihu.com/p/54711962}{重要性质}(实分析 \& 泛函分析):
  \begin{enumerate}
    \item Cantor 集的 Lebesgue 测定是 0
    \item Cantor 集是非空有界闭集
    \item Cantor 集完全集
    \item Cantor 集是无处稠密集(疏朗集)
    \item Cantor 集是不可数集
  \end{enumerate}
\end{anote}


\end{document}
