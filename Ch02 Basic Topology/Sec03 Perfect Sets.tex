\documentclass[../poma-notes.tex]{subfiles}

\graphicspath{\subfix{../images/}}

\begin{document}

\subsection*{Perfect Sets}

\begin{theorem}
  Let $P$ be a noempty perfect set in $R^k$. Then $P$ is uncountable.
\end{theorem}

\begin{proof}
  因为 $P$ 有极限点,$P$ 必须是无限的。假设 $P$ 是可数的,那么通过 $\mathbf{x}_1,\mathbf{x}_2,\mathbf{x}_3,\dots$
  来表示 $P$。随之可以构建一个数列 $\{V_n\}$ 来表示邻域。

  令 $V_1$ 为 $\mathbf{x}_1$ 的任意邻域,使得 $V_n \cap P$ 不为空。如果 $V_1$ 包含了所有的 $\mathbf{y} \in R^k$ 使得
  $|\mathbf{y} - \mathbf{x}_1| < r$,$V_1$ 的闭包 $\overline{V}_1$ 则是所有 $\mathbf{y} \in R^k$ 的集合使得
  $|\mathbf{y} - \mathbf{x}_1| \le r$。

  假设 $V_n$ 被构建了,那么 $V_n \cap P$ 不为空。因为 $P$ 的所有点是 $P$ 的极限点,存在一个邻域 $V_{n+1}$ 使得
  \begin{enumerate*}[label=(\roman*)]
    \item $\overline{V}_{n+1} \subset V_n$,
    \item $\mathbf{x} \notin \overline{V}_{n+1}$,
    \item $V_{n+1} \cap P$ 非空。
  \end{enumerate*}
  根据 (iii),$V_{n+1}$ 满足我们的归纳假设,同时构建可以进行。

  令 $K_n = \overline{V}_n \cap P$。因为 $\overline{V}_n$ 是闭且有界的,$\overline{V}$ 是紧的。因为
  $\mathbf{x}_n \notin K_{n+1}$,$P$ 里没有点在 $\cap_1^{\infty} K_n$ 中。因为 $K_n \subset P$,这意为
  $\cap_1^{\infty} K_n$ 是空的。但是根据 (iii),每个 $K_n$ 是非空的,且根据 (i),$K_n \subset K_{n+1}$;这与 Theorem 2.36
  的 Corollary 相悖。
\end{proof}

\begin{corollary}
  Every interval $[a,b]\ (a<b)$ is uncountable. In particular, the set of all real numbers is uncountable.
\end{corollary}

\end{document}
