\documentclass[../poma-notes.tex]{subfiles}

\graphicspath{\subfix{../images/}}

\begin{document}

\subsection*{Perfect Sets}

\begin{theorem}
  Let $P$ be a nonempty perfect set in $R^k$. Then $P$ is uncountable.
\end{theorem}

\begin{proof}
  因为 $P$ 有极限点,$P$ 必须是无限的。假设 $P$ 是可数的,那么通过 $\mathbf{x}_1,\mathbf{x}_2,\mathbf{x}_3,\dots$
  来表示 $P$。随之可以构建一个数列 $\{V_n\}$ 来表示邻域。

  令 $V_1$ 为 $\mathbf{x}_1$ 的任意邻域,使得 $V_n \cap P$ 不为空。如果 $V_1$ 包含了所有的 $\mathbf{y} \in R^k$ 使得
  $|\mathbf{y} - \mathbf{x}_1| < r$,$V_1$ 的闭包 $\overline{V}_1$ 则是所有 $\mathbf{y} \in R^k$ 的集合使得
  $|\mathbf{y} - \mathbf{x}_1| \le r$。

  假设 $V_n$ 被构建了,那么 $V_n \cap P$ 不为空。因为 $P$ 的所有点是 $P$ 的极限点,存在一个邻域 $V_{n+1}$ 使得
  \begin{enumerate*}[label=(\roman*)]
    \item $\overline{V}_{n+1} \subset V_n$,
    \item $\mathbf{x} \notin \overline{V}_{n+1}$,
    \item $V_{n+1} \cap P$ 非空。
  \end{enumerate*}
  根据 (iii),$V_{n+1}$ 满足我们的归纳假设,同时构建可以进行。

  令 $K_n = \overline{V}_n \cap P$。因为 $\overline{V}_n$ 是闭且有界的,$\overline{V}$ 是紧的。因为
  $\mathbf{x}_n \notin K_{n+1}$,$P$ 里没有点在 $\cap_1^{\infty} K_n$ 中。因为 $K_n \subset P$,这意为
  $\cap_1^{\infty} K_n$ 是空的。但是根据 (iii),每个 $K_n$ 是非空的,且根据 (i),$K_n \subset K_{n+1}$;这与 Theorem 2.36
  的 Corollary 相悖。
\end{proof}

\anote $P$ 为 $R^k$ 内的非空完全集,那么 $P$ 是不可数的。

\begin{corollary}
  Every interval $[a,b]\ (a<b)$ is uncountable. In particular, the set of all real numbers is uncountable.
\end{corollary}

\anote 任何区间都是不可数的,尤其是所有实数的集合都是不可数的。

\begin{namedtheorem}[The Cantor set]
  \normalfont
  The set which we are now going to construct shows that there exist perfect sets in $R^1$ which contain no segment.

  Let $E_0$ be the interval $[0,1]$. Remove the segment $(\frac{1}{3},\frac{2}{3})$, and let $E_1$ be the union of
  the intervals
  \[ [0,\frac{1}{3}]\ [\frac{2}{3},1]. \]
  Remove the middle thirds of these intervals, and let $E_2$ be the union of the intervals
  \[ [0,\frac{1}{9}],\ [\frac{2}{9},\frac{3}{9}],\ [\frac{6}{9},\frac{7}{9}],\ [\frac{8}{9},1]. \]
  Continuing in this way, we obtain a sequence of compact sets $E_n$, such that
  \begin{enumerate}[label=(\alph*)]
    \item $E_1 \supset E_2 \supset E_3 \supset \cdots$;
    \item $E_n$ is the union of $2^n$ intervals, each of length $3^{-n}$.
  \end{enumerate}

  The set
  \[ P = \bigcap\limits_{n=1}^{\infty} E_n \]
  is called the \textit{Cantor set}. $P$ is clearly compact, and Theorem 2.36 shows that $P$ is not empty.

  No segment of the form
  \begin{equation}
    \Biggl( \frac{3k+1}{3^m}, \frac{3k+2}{3^m} \Biggr),
  \end{equation}
  where $k$ and $m$ are positive integers, has a pint in common with $P$. Since every segment $(\alpha, \beta)$
  contains a segment of the form (24), if
  \[ 3^{-m} < \frac{\beta - \alpha}{6}, \]
  $P$ contains no segment.

  To show that $P$ is perfect, it is enough to show that $P$ contains no isolated point. Let $x \in P$, and let $S$
  be any segment containing $x$. Let $I_n$ be that interval of $E_n$ which contains $x$. Choose $n$ large enough,
  so that $I_n \subset S$. Let $x_n$ be an endpoint of $I_n$, such that $x_n \ne x$.

  It follows from the construction of $P$ that $x_n \in P$. Hence $x$ is a limit point of $P$, and $P$ is perfect.

  One of the most interesting properties of the Cantor set is that it provides us with an example of an uncountable
  set of measure zero (the concept of measure will be discussed in Chap. 11).
\end{namedtheorem}

\anote Cantor 集说明了 $R^1$ 中存在没有区间的完全集。

\end{document}
