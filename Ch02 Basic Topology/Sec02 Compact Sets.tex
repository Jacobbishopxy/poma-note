\documentclass[../poma-notes.tex]{subfiles}

\graphicspath{\subfix{../images/}}

\begin{document}

\subsection*{Compact Sets}

\begin{definition}
  By an \textit{open cover} of a set $E$ in a metric space $X$ we mean a collection $\{G_{\alpha}\}$ of open subsets
  of $X$ such that $E \subset \cup_{\alpha} G_{\alpha}$.
\end{definition}

\anote \textit{开覆盖} open cover。

\begin{definition}
  A subset $K$ of a metric space $X$ is said to be \textit{compact} if every open cover of $K$ contains a $finite$
  subcover.

  More explicitly, the requirement is that if $\{G_{\alpha}\}$ is an open cover of $K$, then there are finitely many
  indices $\alpha_1,\dots,\alpha_n$ such that
  \[K \subset G_{\alpha_1} \cup \cdots \cup G_{\alpha_n}\ .\]
\end{definition}

紧凑 compactness 这个概念在数学分析中非常的重要,特别是它连接了连续性这个概念(第四章)。

很明显所有的有限集都是紧的。$R^k$ 上存在大类无限紧集将会在 Theorem 2.14 中提及。

我们之前观察到的(Remark 2.29)如果 $E \subset Y \subset X$,那么 $E$ 有可能相对 $Y$ 是开的,而不需要相对 $X$ 是开的。
开这个属性因此取决于 $E$ 所嵌入的空间,对于闭这个属性亦是如此。

\begin{anote}
  \begin{itemize}
    \item 紧集 compact set:任何开覆盖都存在有限的子覆盖。
    \item 若一个集合是紧集,就可以说这个集合具有紧性 compacted。
    \item 紧性实际上是一种拓扑性质。
  \end{itemize}
\end{anote}

\begin{theorem}
  Suppose $K \subset Y \subset X$. Then $K$ is compact relative to $X$ if and only if $K$ is compact relative to $Y$.
\end{theorem}

凭借这个定理,在许多情况下,我们能够将紧集本身视为度量空间,而无需关注任何嵌入空间。尽管讨论\textit{开}空间,或者\textit{闭}
空间的意义不大(每个度量空间 $X$ 是其自身的一个开子集,以及一个闭子集),但是讨论\textit{紧}度量空间却是很有意义的。

\begin{proof}
  假设 $K$ 是相对 $X$ 紧的,且令 $\{V_{\alpha}\}$ 为相对 $Y$ 是开的集合类,使得 $K \subset \cup_{\alpha} V_{\alpha}$。
  根据 Theorem 2.30,存在若干相对 $X$ 是开的集合 $G_{\alpha}$,使得对于所有 $\alpha$ 有 $V_{\alpha} = Y \cap G_{\alpha}$;
  又因为 $K$ 是相对 $X$ 紧的,我们有
  \begin{equation}
    K \subset G_{\alpha_1} \cup \cdots \cup G_{\alpha_n}
  \end{equation}
  对于一些有限数量索引 $\alpha_1, \dots, \alpha_n$。因为 $K \subset Y$,(22) 意味着
  \begin{equation}
    K \subset V_{\alpha_1} \cup \cdots \cup V_{\alpha_n}
  \end{equation}
  这证明了 $K$ 是相对 $Y$ 紧的。

  相反的,假设 $K$ 是相对 $Y$ 紧的,令 $\{G_{\alpha}\}$ 为 $X$ 开子集的类,其覆盖了 $K$,且令 $V_{\alpha}=Y\cap G_{\alpha}$。
  那么 (23) 将包含一些 $\alpha_1, \dots, \alpha_n$;又因为 $V_{\alpha} \subset G_{\alpha}$,(23) 意味着 (22)。

  证明结束。
\end{proof}

\anote 证明流程与 Theorem 2.30 类似,且使用了其结论。

\begin{theorem}
  Compact subsets of metric spaces are closed.
\end{theorem}

\begin{proof}
  令 $K$ 为度量空间 $X$ 的一个紧子集。我们应该证明 $K$ 的补集是 $X$ 上的一个子开集。

  假设 $p \in X,\ p \notin K$。如果 $q \in K$,令 $V_q$ 与 $W_q$ 分别为 $p$ 与 $q$ 的邻域,它们的半径小于 $\frac{1}{2}d(p,q)$
  (详见 Definition 2.18 (a))。由于 $K$ 是紧的,那么在 $K$ 中存在有限数量的点 $q_1,\dots,q_n$ 使得
  \[K \subset W_{q_1} \cup \cdots \cup W_{q_n} = W\ .\]
  如果 $V = V_{q_1} \cup \cdots \cup V_{q_n}$,那么 $V$ 是 $p$ 的一个邻域,且不与 $W$ 相交。因此 $K \subset K^c$,所以 $p$ 是
  $K^c$ 的一个内点。
\end{proof}

\anote 度量空间的紧子集是闭的;度量空间的紧集都是有界的.

\begin{theorem}
  Closed subsets of compact sets are compact.
\end{theorem}

\begin{proof}
  假设 $F \subset K \subset X$,$F$ 是闭的(相对于 $X$),且 $K$ 是紧的。令 $\{V_{\alpha}\}$ 为 $F$ 的一个开覆盖。如果 $F^c$
  临近 $\{V_{\alpha}\}$,我们得到一个 $K$ 的开覆盖 $\Omega$。由于 $K$ 是紧的,那么存在一个有限的子类 $\Phi$ 覆盖 $K$,也覆盖了
  $F$。如果 $F^c$ 是 $\Phi$ 的一个成员,我们将其从 $Phi$ 中移除并保持 $F$ 的一个开覆盖。综上所述,$\{V_{\alpha}\}$ 的一个有限的
  子类覆盖了 $F$。
\end{proof}

\anote 紧集的子闭集是紧的。

\begin{corollary}
  If $F$ is closed and $K$ is compact, then $F \cap K$ is compact
\end{corollary}

\begin{proof}
  Theorem 2.24 (b) 以及 2.34 展示了 $F \cap K$ 是闭的;由于 $F \cap K \subset K$。Theorem 2.35 展示了 $F \cap K$ 是紧的。
\end{proof}

\anote 闭集与紧集的交集是紧的。

\begin{theorem}
  If $\{K_{\alpha}\}$ is a collection of compact subsets of a metric space $X$ such that the intersection of every finite
  subcollection of $\{K_{\alpha}\}$ is nonempty, then $\cap K_{\alpha}$ is nonempty.
\end{theorem}

\begin{proof}
  从 $\{K_{\alpha}\}$ 中固定一个成员 $K_1$ 并令 $G_{\alpha} = K_{\alpha}^c$。假设 $K_1$ 中没有点属于所有的 $K_{\alpha}$。
  那么集合 $G_{\alpha}$ 形成了 $K_1$ 的一个开覆盖;又因为 $K_1$ 是紧的,存在有限数量的索引 $\alpha_1,\dots,\alpha_n$ 使得
  $K_1 \subset G_{\alpha_1} \cup \cdots \cup G_{\alpha_n}$。但是这就意味着
  \[K_1 \cap K_{\alpha_1} \cap \cdots \cap K_{\alpha_n}\]
  为空,与假设相悖。
\end{proof}

\begin{corollary}
  If $\{K_n\}$ is a sequence of nonempty compact sets such that $K_n \supset K_{n+1} \ (n=1,2,3,\dots)$, then
  $\cap_1^{\infty} K_n$ is not empty.
\end{corollary}

\anote
如果 $\{K_n\}$ 是一个非空紧集的数列,使得 $K_n \supset K_{n+1} \ (n=1,2,3,\dots)$,那么 $\cap_1^{\infty} K_n$ 不为空。

\end{document}
