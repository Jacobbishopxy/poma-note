\documentclass[../poma-notes.tex]{subfiles}

\graphicspath{\subfix{../images/}}

\begin{document}

\subsection*{Compact Sets}

\begin{definition}
  By an \textit{open cover} of a set $E$ in a metric space $X$ we mean a collection $\{G_{\alpha}\}$ of open subsets
  of $X$ such that $E \subset \cup_{\alpha} G_{\alpha}$.
\end{definition}

\anote \textit{开覆盖} open cover。

\begin{definition}
  A subset $K$ of a metric space $X$ is said to be \textit{compact} if every open cover of $K$ contains a $finite$
  subcover.

  More explicitly, the requirement is that if $\{G_{\alpha}\}$ is an open cover of $K$, then there are finitely many
  indices $\alpha_1,\dots,\alpha_n$ such that
  \[K \subset G_{\alpha_1} \cup \cdots \cup G_{\alpha_n}\ .\]
\end{definition}

紧凑 compactness 这个概念在数学分析中非常的重要,特别是它连接了连续性这个概念(第四章)。

很明显所有的有限集都是紧凑的。$R^k$ 上存在大类无限紧凑集将会在 Theorem 2.14 中提及。

我们之前观察到的(Remark 2.29)如果 $E \subset Y \subset X$,那么 $E$ 有可能相对 $Y$ 是开的,而不需要相对 $X$ 是开的。
开这个属性因此取决于 $E$ 所嵌入的空间,对于闭这个属性亦是如此。

\begin{anote}
  \begin{itemize}
    \item 紧集 compact set:任何开覆盖都存在有限的子覆盖。
    \item 若一个集合是紧集,就可以说这个集合具有紧性 compacted。
    \item 紧性实际上是一种拓扑性质。
  \end{itemize}
\end{anote}

\begin{theorem}
  Suppose $K \subset Y \subset X$. Then $K$ is compact relative to $X$ if and only if $K$ is compact relative to $Y$.
\end{theorem}

凭借这个定理,在许多情况下,我们能够将紧集本身视为度量空间,而无需关注任何嵌入空间。尽管讨论\textit{开}空间,或者\textit{闭}
空间的意义不大(每个度量空间 $X$ 是其自身的一个开子集,以及一个闭子集),但是讨论\textit{紧}度量空间却是很有意义的。

% TODO
% \begin{proof}
% \end{proof}

\end{document}
